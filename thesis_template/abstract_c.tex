
%\begin{CJK}{UTF8}{bsmi}  

    平均曲率流首先由Mullins\cite{mullins_twodimensional_1956}公式化,描述了沿其平均曲率矢量移動的曲面的演變。在具有邊界的曲面的平均曲率流中,具有自由邊界的平均曲率流受到了最廣泛的研究。這類平均曲率流非常重要,並且與自由邊界最小曲面和液體界面的受約束運動密切相關。

本文旨在為黎曼流形中具有自由邊界的平均曲率流的收斂理論提供理論基礎。通過回顧Huisken\cite{huisken_flow_1984}的經典方法,我們強調了迭代方案對於證明無跡第二基本形式的收縮估計的重要性。接下來,我們利用Andrews和Baker\cite{andrews_mean_2010}引入的平均曲率流的協變量公式,計算第二基本形式的邊界導數。我們還將\cite[Theorem 3.1]{edelen_convexity_2016}推廣到黎曼流形作為環境空間的迭代方案,該迭代方案用於證明滿足兩個特殊不等式的函數的統一界。本文的結果有助於將Hirsch與Li\cite{hirsch2020contracting}的收斂結果推廣到黎曼流形中的自由邊界平均曲率流。
    
%\end{CJK}

