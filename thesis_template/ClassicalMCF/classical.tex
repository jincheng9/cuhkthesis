%
% chapter conclusion

\chapter{Classical Results of MCF} \label{chap:crmcf}

This chapter outlines a general methodology proposed by Huisken~\cite{huisken_flow_1984} for proving the convergence of surfaces to a round point. Following the similar idea, mathematicians~\cite{andrews_mean_2010,liu2012mean,hirsch2020contracting} have generalized the convergence theorem to different conditions. Therefore, it is essential to review Huisken's~\cite{huisken_flow_1984} classical arguments and capture the idea behind them.

Throughout this chapter, we let $M$ be a compact uniformly convex hypersurface smoothly embedded in $\R^{n+1}$. Any such $M$ can be represented locally by the following diffeomorphism:
\[
	F\colon U \subset \R^n \rightarrow M \subset \R^{n+1}.
\]
The metric $g= \left\{ g_{ij} \right\}  $ and the second fundamental form $A= \left\{ h_{ij} \right\}  $ at $F(\vec{x}) \in M$ can be written as
\[
	g_{ij}(\vec{x})=\left( \frac{\partial F(\vec{x})}{\partial x_i}, \frac{\partial F(\vec{x})}{\partial x_j} \right) ,\quad h_{ij}(\vec{x})=\left( -\nu (\vec{x}), \frac{\partial^2 F(\vec{x})}{\partial x_i \partial x_j} \right)
\]
where $\nu (\vec{x}) \in \R^{n+1}$ is the outward normal to $M$ at $F(\vec{x})$ and $\left( \cdot , \cdot \right) $ is the standard inner product in $\R^{n+1}$. The Levi-Civita connection on $M$ induced from the standard connection on $\R^{n+1}$ is given by 
\[\Gamma_{\ ij}^{k} =\frac{1}{2}g_{ }^{kl} \left( g_{ il,j}^{} +g_{ jl,i}^{} - g_{ij ,l}^{}  \right) \]
where $g_{ij ,k}=\frac{\partial }{\partial x_{k}} g_{ij}$. For a vector field $X=X^i \frac{\partial }{\partial x_{i}} $ on $M$, the covariant derivative of $X$ is \[\left( \nabla_{i}^{} X \right) ^j=\frac{\partial }{\partial x_{i}} X^j + \Gamma_{\ ik}^{j} X^k.\]
The Riemann curvature tensor on $M$ is defined as 
\[R_{ijkl}= \left< (\nabla_{i}^{} \nabla_{j}^{} -\nabla_{j}^{} \nabla_{i}^{}) \frac{\partial }{\partial x_{k}} , \frac{\partial }{\partial x_{l}}  \right> \] where $\left\langle  \cdot , \cdot \right\rangle $ is the inner product for tensors on $M$ induced from $g$. By Gauss equation, we have that 
\begin{equation} \label{eq:gauss}
	R_{ijkl}=h_{ik}h_{jl }^{}-h_{ il}^{} h_{ jk}^{} .
\end{equation}
The Ricci tensor and scalar curvature are thus given by
\[R_{ik}=Hh_{ik}-h_{ i}^{\ j} h_{ jk}^{} , \quad R=H^2-\left| A \right| ^2\]
where $H=g^{ij}h_{ij}$, $\left| A \right| ^2=h^{ij}h_{ij}$ and the metric tensor $g$ is used to raise or lower indices.

Now we denote $M$ by $M_0$ and $F$ by $F_0$. We say a family of maps $F(\cdot, t)$ satisfies the MCF equation with initial condition $F_0$ if
 \begin{equation}\label{eq:evo}
 \begin{split}
	& \frac{\partial }{\partial t} F(\vec{x},t)=-H(\vec{x},t) \cdot \nu (\vec{x},t), \quad \vec{x} \in U, \\
	& F(\cdot ,0)=F_0,
 \end{split}
 \end{equation} 
where $H(\vec{x},t)$ is the mean curvature on $M_t$.

To prove that $M_t$ converges to a round point as $t \to T$, we first show that the hypersurface $M_t$ converges to a point. Then we normalize the flow by keeping the total area of $M_t$ fixed and prove that $M_t$ converges to a round sphere under the normalized flow.

For the second part, we rescale the solution $F$ at each time $t \in [0,T)$ by a positive constant $\psi(t)$ such that 
\begin{equation} \label{fixarea}
	\int_{\tilde{M}_t}^{} d \tilde{\mu }_t = \left| M_0 \right| \text{ for all } 0 \leq t < T
\end{equation}
where the surface $\tilde{M}_t$ is given by local diffeomorphisms
\[\tilde{F}(\cdot ,t)=\psi (t) \cdot F(\cdot ,t).\]
By introducing a new time variable $\tilde{t}(t)=\int_{0}^{t}\psi ^2(\tau )d \tau $, we derive in \autoref{sec:crp} that $\tilde{F}$ satisfies the following normalized equation on a different maximal time interval $\tilde{t} \in [0,\tilde{T})$:
\begin{equation} \label{eq:reevo}
	\frac{\partial \tilde{F}}{\partial \tilde{t}}=-\tilde{H}\tilde{\nu }+ \frac{1}{n}\tilde{h}\tilde{F}
\end{equation}
where $\tilde{h}=\frac{\int_{}^{}\tilde{H}^2 d \tilde{\mu} }{\int_{}^{}d \tilde{\mu} }$ is the average of the squared mean curvature on $\tilde{M}_t$.
Now we can state the convergence theorem by Huisken.
\begin{theorem}[{\cite[Theorem 1.1]{huisken_flow_1984}}] \label{thm:main1}
	Under the assumption that $n \geq 2$ and $M_0$ is uniformly convex, \autoref{eq:evo} has a smooth solution on $[0,T)$ for $T<\infty $, and $M_t$ converges to a point as $t \to T$. The solution to the normalized \autoref{eq:reevo} exists for all time $0 \leq \tilde{t}< \infty$ and converges to a sphere in the $C^{\infty }$-topology as $\tilde{t} \to \infty $.  
\end{theorem}

\section{Maximum principles and Simons' identity}

Parabolic maximum principles are essential PDE tools in the analysis of MCF. We will briefly introduce two frequently used versions in this section. One is the standard parabolic maximum principle for scalar functions.

\begin{theorem}[{\cite[p.397]{evans_partial_2010}}] \label{thm:MPP}
	Let $M$ be a closed smooth manifold and \linebreak ${f\colon M \times [0,T) \to \R}$ be a scalar function on $M$ varying along time $t$. Suppose $f (\cdot,0) \geq 0$ and
	\[
		\frac{\partial f}{\partial t} \geq \Delta f + b^{i} \nabla_{i}^{} f+ cf
	\]
	for some smooth function $b^i,c$, where $c \geq 0.$ Then
	\[\min _M f (\cdot,t) \geq \min _M f (\cdot,0).\]
	Furthermore, if there exist some $p \in M$ and $t_0 \geq 0$ such that $f(p,t_0)=\min _M f (\cdot,0) $, then $f \equiv \min _M f (\cdot,0)$ for $0 \leq t \leq t_0.$
\end{theorem}
Later we will see that the mean curvature of the evolving hypersurface satisfies such a parabolic inequality. By the strong maximum principle, we can show that the positivity of $H$ is preserved throughout the flow.

In order to study the evolution of tensors such as the second fundamental form, we extend the scalar maximum principle to tensors. Let $M_{ij}$ be a symmetric tensor on a closed manifold $M$. We say $M_{ij }^{} \geq 0 $ if for any vector $X$ on $M$, $M_{ij }^{} X^i X^j \geq 0$. Let $N_{ij }^{} = P(M_{ij }^{} , g_{ij}^{} )$ be another symmetric tensor formed by contracting $M_{ij }^{} $ with itself using the metric where $P$ is a polynomial. Then we can prove a weak maximum principle for symmetric $2$-tensors by following the treatment in \cite[Theorem 4.6]{chow_ricci_2004}.

\begin{theorem}[{\cite[Theorem 9.1]{hamilton_three-manifolds_1982}}] \label{thm:MP2T}
	Suppose $M_{ij }^{} $ is a symmetric tensor on a closed manifold $M$ depending on time $t$ and on $0 \leq t < T$ satisfies that \[\frac{\partial }{\partial t} M_{ij }^{} = \Delta M_{ij }^{} + u^k \nabla_{k}^{} M_{ij }^{} + N_{ij }^{} \] where $u^k$ is a vector on $M$ and $N_{ij }^{} $ is defined as above such that 
	\[
		N_{ij }^{} X^i X^j \geq 0 \text{   whenever  } M_{ij }^{} X^j =0.
	\]
	Then if $M_{ij }^{} \geq 0$ at $t=0$, it will remain so on $0 \leq t < T$.
\end{theorem}

\begin{proof}
	
	Given any $\tau \in  [0,T)$, we claim that there exists $\delta >0$ such that for any $t_0 \in [0,\tau -\delta]$, if $M_{ij }^{} \geq 0$ when $t=t_0$, then $M_{ij }^{} \geq 0$ for $t \in [t_0,t_0+\delta ]$. The theorem follows naturally from the claim.
	
	Now we fix some $t_0 \in [0,\tau -\delta]$ and let $\tilde{M}_{ij }^{}  = M_{ij }^{} + \epsilon (\delta +t-t_0) g_{ij}^{} $ for $0<\epsilon \leq 1$.Note that when $t=t_0$, $\tilde{M}_{ij }^{}  = M_{ij }^{} + \epsilon \delta g_{ij}^{} >M_{ij }^{}$. To prove the claim, we only need to show that $\tilde{M}_{ij }^{}>0$ on $(t_0,t_0+\delta ]$ for $\delta $ independent of $\epsilon $ and let $\epsilon \to 0$.
	
	Suppose the statement were not true. Then $\tilde{M}_{ij }^{}$ has null eigenvectors for a first time $t_1 \in (t_0,t_0+\delta]$ at $x_1 \in M_{t_1}$. Let $X=X^i \frac{\partial }{\partial x_{i}} \in T_{x_1}M_{t_1}$ be a unit null eigenvector of $\tilde{M}_{ij }^{}$. We set $\tilde{N}_{ij }^{} =P(\tilde{M}_{ij }^{} , g_{ij }^{} )$. Note that $N_{ij }^{} = P(M_{ij }^{} , g_{ij}^{} )$ have the property that  ${N}_{ij }^{} Y^i Y^j \geq 0$ whenever $Y$ is a null eigenvector of $M_{ij }^{} $. Since $X$ is a null eigenvector of $\tilde{M}_{ij }^{}$, we have that $\tilde{N}_{ij }^{} X^i X^j \geq 0$. Then at $(x_1,t_1)$, 
	\begin{equation*}
	\begin{split}
		N_{ij }^{} X^i X^j
	&= \tilde{N}_{ij }^{} X^i X^j + (N_{ij }^{} - \tilde{N}_{ij }^{}) X^i X^j\\
	& \geq (N_{ij }^{} - \tilde{N}_{ij }^{}) X^i X^j\\
	& \geq -\left| N_{ij }^{} - \tilde{N}_{ij }^{} \right|.
	\end{split}
	\end{equation*}
	Since $P$ is a polynomial, we have that \[\left| N_{ij }^{} - \tilde{N}_{ij }^{} \right| \leq C \left| M_{ij }^{} - \tilde{M_{ij }^{}} \right|\] where $C$ is a constant depending only on $\max_{M \times [0,\tau ]} \left| M_{ij }^{}  \right|. $ As $t_1 \in [t_0,t_0+\delta ]$, we have that at $(x_1,t_1)$
	\begin{equation*}
	\begin{split}
		N_{ij }^{} X^i X^j 
		& \geq -C \left| M_{ij }^{} - \tilde{M_{ij }^{}} \right|\\
		&= -C \left| \epsilon (\delta + t_1-t_0) g_{ij }^{}  \right|\\
		& \geq -2C \epsilon \delta.
	\end{split}
	\end{equation*}

	The point $(x_1,t_1)$ depends on the time $t_0$ we choose in $[0,\tau]$. Nevertheless, we can choose $\delta_0>0$ sufficiently small depending only on $\max_{M \times [0,\tau ]} \left| \frac{\partial }{\partial t} g_{ij }^{}  \right|$ such that at $(x_1,t_1)$,
	\[\frac{\partial }{\partial t} g_{ij }^{} \geq -\frac{1}{\delta_0 }g_{ij }^{} .\]

	The evolution of $\tilde{M}_{ij }$ is given by 
	\[\frac{\partial }{\partial t} \tilde{M}_{ij }^{} = \frac{\partial }{\partial t} M_{ij }^{} + \epsilon g+\epsilon (\delta +t-t_0) \frac{\partial }{\partial t}  g_{ij}^{}.\]
	Since $\Delta \tilde{M}_{ij}=\Delta M_{ij}$ and $\nabla \tilde{M}_{ij}=\nabla M_{ij}$, then at $(x_1,t_1)$,  
	\begin{equation*}
	\begin{split}
		\frac{\partial }{\partial t} \tilde{M}_{ij }^{} &\geq \Delta M_{ij }^{} + u^k \nabla_{k}^{} M_{ij }^{}+ N_{ij }^{}+ \epsilon g+\epsilon (\delta +t_1-t_0) \frac{\partial }{\partial t}  g_{ij}^{}\\
	&\geq \Delta \tilde{M}_{ij }^{}+ u^k \nabla_{k}^{} \tilde{M}_{ij }^{} + N_{ij }^{} + (1-\frac{2 \delta }{\delta_0})\epsilon g_{ij }^{}. 
	\end{split}
	\end{equation*}

	By extending the vector $X^i$ to a parallel vector field in a neighborhood of $x_1$ along geodesics passing $x_1$ on $M_{t_1}$ and defining $X^i$ on $[t_0,t_0+\delta]$ independent of $t$, we can define a function $f=\tilde{M}_{ij }^{} X^i X^j$ on $M \times [t_0,t_0+\delta]$. By assumption, we have that $f (x_1,t) > 0 $ for $t_0 \leq t<t_1$ and $f ( x_1,t_1) =0$. Hence $\frac{\partial }{\partial t} f(x_1,t_1) \leq 0$. At $t=t_1$, we see that $f=0$ attains a minimum at $x_1$; thus $\nabla f(x_1,t_1)=0$ and $\Delta f(x_1,t_1) \geq 0$.
	Moreover, since $X$ is parallel, we have that
	\begin{align*}
		\frac{\partial }{\partial t} f&=(\frac{\partial }{\partial t} \tilde{M}_{ij }^{}) X^i X^j\\
		\nabla_k f&= (\nabla_k \tilde{M}_{ij }^{}) X^i X^j \\
		\Delta f &= (\Delta  \tilde{M}_{ij }^{}) X^i X^j.
	\end{align*}

	Therefore, at $(x_1,t_1)$,
	\begin{equation*}
	\begin{split}
		0>\frac{\partial }{\partial t} f
		&=(\frac{\partial }{\partial t} \tilde{M}_{ij }^{}) X^i X^j\\
		&\geq \left[ \Delta \tilde{M}_{ij }^{}+ u^k \nabla_{k}^{} \tilde{M}_{ij }^{} + N_{ij }^{} + (1-\frac{2 \delta }{\delta_0})\epsilon g_{ij }^{} \right] X^i X^j\\
		&= \Delta f + u^k \nabla_{k}^{} f + (1-\frac{2 \delta }{\delta_0}-2C \delta ) \epsilon\\
		&\geq (1-\frac{2 \delta }{\delta_0}-2C \delta ) \epsilon.
	\end{split}
	\end{equation*}
	Then contradiction arises when $\delta <\frac{1}{4} \min\{\delta _0, \frac{1}{C}\}.$ Since now $\delta $ depends only on $\max_{M \times [0,\tau ]} \left| M_{ij }^{}  \right|$ and $\max_{M \times [0,\tau ]} \left| \frac{\partial }{\partial t} g_{ij }^{}  \right|$, we can let $\epsilon \to 0$ and reach the conclusion. 

\end{proof}

To apply the maximum principles, we need the following Simons' identity to rewrite the evolution equation of the second fundamental form as a parabolic PDE.

\begin{lemma}
	[Simons' identity] \label{simon}
	\[\Delta h_{ij}^{} = \nabla_{i}\nabla_{j} H + H h_{li}^{} g_{}^{lm } h_{mj}^{} - \left| A \right| ^2 h_{ij}^{} \]
\end{lemma}

\begin{proof}
	Note that $\Delta h_{ij}^{} = g_{}^{mn} \nabla_{m}\nabla_{n} h_{ij}^{} $ and $\nabla_{i}\nabla_{j} H = g_{}^{mn } \nabla_{i}\nabla_{j} h_{mn }^{} $. It suffices to examine the difference $\nabla_{m}\nabla_{n} h_{ij}^{} - \nabla_{i}\nabla_{j} h_{mn}^{} .$ Since the ambient space is Euclidean, from the Codazzi equation we have that $\nabla_{i}^{} h_{j}^{k} = \nabla_{j}^{} h_{i}^{k}$. Hence \[\nabla_{m}\nabla_{n} h_{ij}^{} - \nabla_{i}\nabla_{j} h_{mn}^{} = \nabla_{m}\nabla_{i} h_{nj}^{} - \nabla_{i}\nabla_{m} h_{jn}^{}=(\nabla_{m}\nabla_{i}  - \nabla_{i}\nabla_{m})h_{nj}^{}.\] By the product rule of connections acting on tensor product, we have that \[(\nabla_{m}\nabla_{i}  - \nabla_{i}\nabla_{m})h_{nj}^{} = R_{min}^{\quad\  l} h_{lj}^{} + R_{mij}^{\quad \ l} h_{nl }^{}.  \]
	Therefore, by \autoref{eq:gauss}
	\begin{equation*}
		\begin{split}
			\Delta h_{ij}^{} - \nabla_{i}\nabla_{j} H
			&= g_{}^{mn} (R_{min }^{\quad \ l} h_{lj}^{} + R_{mij}^{\quad \ l} h_{nl }^{} ) \\
			&= g_{}^{mn } g_{}^{kl} \left\{ \left( h_{mn}^{} h_{ik }^{} -h_{mk }^{} h_{in }^{}  \right)  h_{lj}^{} + (h_{mj }^{} h_{ik }^{} - h_{mk }^{} h_{ij}^{} )h_{ln}^{} \right\} \\
			&= H g_{}^{kl} h_{ik }^{}  h_{lj}^{} - g_{}^{mn } g_{}^{kl } h_{mk }^{} h_{ln }^{} h_{ij}^{} \\
			&= H g_{}^{kl} h_{ik }^{}  h_{lj}^{} - \left| A \right| ^2 h_{ij}^{} .
		\end{split}
	\end{equation*}
\end{proof}


\section{Evolution equations for geometric quantities}


Since the embedding map $F$ is evolving under time $t$, if we fix a point $\vec{x} \in U$, geometric quantities at $F(\vec{x},t) \in M_t$ are also evolving under time $t$. By the flow equation $\frac{\partial }{\partial t} F(\vec{x},t)=-H(\vec{x},t) \cdot \nu (\vec{x},t)$, we can derive evolution equations for other geometric quantities.

\begin{lemma}[{\cite[Section 3]{huisken_flow_1984}}]
	The following evolution equations hold.
	\begin{enumerate}[\normalfont(1)]

		\item $\frac{\partial }{\partial t} g_{ij}=-2Hh_{ij}$
		\item $\frac{\partial }{\partial t} g^{ij}=2Hh^{ij}$
		\item $\frac{\partial \nu }{\partial t} = \nabla H$
		\item $\frac{\partial }{\partial t} h_{ij}=\Delta h_{ij}-2Hh_{ik}g^{kl}h_{lj}+\left| A \right| ^2 h_{ij}$
		\item $\frac{\partial }{\partial t} H=\Delta H+ \left| A \right| ^2 H$
		\item $\frac{\partial }{\partial t} \left| A \right| ^2 = \Delta \left| A \right| ^2 - 2 \left| \nabla A \right| ^2 + 2 \left| A \right| ^4$
	\end{enumerate}
\end{lemma}

\begin{proof}\leavevmode
	\begin{enumerate}[\normalfont(1)]
		\item Since $\left( \nu ,\frac{\partial F}{\partial x_{i}}  \right) =0$, by the product rule, we have that
		      \begin{equation*}
			      \begin{split}
				      \frac{\partial }{\partial t} g_{ij}^{}
				      & = \frac{\partial }{\partial t} \left( \frac{\partial F(\vec{x},t) }{\partial x_{i}} , \frac{\partial F(\vec{x},t) }{\partial x_{j }}  \right)                                                                                                               \\
				      & = \left( \frac{\partial }{\partial x_{i}} (-H (\vec{x},t) \cdot \nu (\vec{x},t)) , \frac{\partial F}{\partial x_{j}} ) \right) \\
					  & \quad + \left( \frac{\partial F}{\partial x_{i}}, \frac{\partial }{\partial x_{j}} (-H (\vec{x},t) \cdot \nu (\vec{x},t))  \right) \\
				      & = -H \left[ \left( \frac{\partial \nu }{\partial x_{i}} , \frac{\partial F}{\partial x_{j}}  \right) + \left( \frac{\partial F}{\partial x_{i}} , \frac{\partial \nu }{\partial x_{j}}  \right) \right]                                                                    \\
				      & = -2H h_{ij}^{}.
			      \end{split}
		      \end{equation*}
		\item Since $g_{km}^{} g_{}^{mj} = \delta_{k}^{j} $, we have that
		      \begin{align*}
			      \frac{\partial }{\partial t} (g_{km}^{} g_{}^{mj})                                                  & =0                                \\
			      \frac{\partial g_{km}^{} }{\partial t} g_{}^{mj} + g_{km}^{} \frac{\partial g_{}^{mj} }{\partial t} & =0                                \\
			      -2H h_{km}^{} g_{}^{mj} + g_{km}^{} \frac{\partial g_{}^{mj} }{\partial t}                          & =0                                \\
			      g_{}^{ik} g_{km}^{} \frac{\partial g_{}^{mj} }{\partial t}                                          & =g_{}^{ik} 2H h_{km}^{} g_{}^{mj} \\
			      \frac{\partial }{\partial t} g_{}^{ij}                                                              & = 2H h_{}^{ij}.
		      \end{align*}
		\item Since $\left| \nu \right| =1$ is fixed, we have that $\frac{\partial \nu }{\partial t} $ lies in the tangent space of the surface. Hence we can assume that $\frac{\partial \nu }{\partial t} = V^i \frac{\partial F}{\partial x_{i}} \in \R^{n+1}$ where $V^i$ can be determined by the following identity \[\left( \frac{\partial \nu }{\partial t} ,\frac{\partial F}{\partial x_{j}}  \right) = g_{ij}^{} V^i.\] Thus, we have that
		      \begin{equation*}
			      \begin{split}
				      \frac{\partial \nu }{\partial t} &=g_{}^{ij} \left( \frac{\partial \nu }{\partial t} ,\frac{\partial F}{\partial x_{j}}  \right) \cdot \frac{\partial F}{\partial x_{i}} \\
				      &= -g_{}^{ij} \left( \nu  ,\frac{\partial }{\partial t} \frac{\partial F}{\partial x_{j}}  \right) \cdot \frac{\partial F}{\partial x_{i}} \\
				      &= g_{}^{ij} \left( \nu  , \frac{\partial }{\partial x_{j}} (H (\vec{x},t) \cdot \nu (\vec{x},t) ) \right) \cdot \frac{\partial F}{\partial x_{i}} \\
				      &= g_{}^{ij} \frac{\partial H}{\partial x_{j}}  \frac{\partial F}{\partial x_{i}} \\
				      &=\nabla H.
			      \end{split}
		      \end{equation*}
		\item By the Gauss-Weingarten relations, we have that
		      \[
			      \begin{cases}
				      \frac{\partial ^2 F}{\partial x_{i} \partial x_{j}} = \Gamma_{\ ij}^{k} \frac{\partial F}{\partial x_{k}} -h_{ij}^{} \nu \\
				      \frac{\partial \nu }{\partial x_{j}} =h_{jl}^{} g_{}^{lm } \frac{\partial F}{\partial x_{m}} .
			      \end{cases}
		      \]
		      Hence
		      \begin{equation*}
			      \begin{split}
				      \frac{\partial }{\partial t} h_{ij}^{}  &= -\frac{\partial }{\partial t} \left( \nu , \frac{\partial^2 F}{\partial x_{i} \partial x_{j}} \right)  \\
				      &= -\left( g_{}^{pq } \frac{\partial H}{\partial x_{p}}  \frac{\partial F}{\partial x_{q}} , \frac{\partial^2 F}{\partial x_{i} \partial x_{j}}  \right) + \left( \nu , \frac{\partial^2 }{\partial x_{i} \partial x_{j}} (H \cdot \nu ) \right) \\
				      &= -\left( g_{}^{pq } \frac{\partial H}{\partial x_{p}}  \frac{\partial F}{\partial x_{q}} , \Gamma_{\ ij}^{k} \frac{\partial F}{\partial x_{k}} -h_{ij}^{} \nu  \right)\\
					  &\quad+ \frac{\partial }{\partial x_{j}} \left( \nu , \frac{\partial }{\partial x_{i}} (H \cdot \nu ) \right) - \left( h_{jl}^{} g_{}^{lm } \frac{\partial F}{\partial x_{m}}, \frac{\partial }{\partial x_{i}} (H \cdot \nu ) \right) \\
				      &= -g_{}^{pq } \frac{\partial H}{\partial x_{q}} \Gamma_{\ ij}^{k} g_{pk}^{} + \frac{\partial^2 H}{\partial x_{i} \partial x_{j}} \\
					  &\quad-H \cdot \left( h_{jl}^{} g_{}^{lm } \frac{\partial F}{\partial x_{m}}, h_{il'}^{} g_{}^{l'm' } \frac{\partial F}{\partial x_{m'}} \right) \\
				      &= \frac{\partial^2 H}{\partial x_{i} \partial x_{j}} - \Gamma_{\ ij}^{q} \frac{\partial H}{\partial x_{q}} - H h_{j}^{m} h_{i}^{n} g_{mn }^{}.
			      \end{split}
		      \end{equation*}
		      Since $H$ is a scalar function, we have that \[\nabla_{i}\nabla_{j} H = \frac{\partial^2 H}{\partial x_{i} \partial x_{j}} - \Gamma_{\ ij}^{q} \frac{\partial H}{\partial x_{q}}\] where $\nabla $ is the Levi-Civita connection on $M_t$. 
			  
			Hence, by \autoref{simon},
		      \begin{equation*}
			      \begin{split}
				      \frac{\partial }{\partial t} h_{ij}^{}
				      &= \frac{\partial^2 H}{\partial x_{i} \partial x_{j}} - \Gamma_{\ ij}^{q} \frac{\partial H}{\partial x_{q}} - H h_{j}^{m} h_{i}^{n} g_{mn }^{} \\
				      &= \Delta h_{ij}^{} -( H h_{li }^{} g_{}^{lm } h_{mj}^{} - \left| A \right| ^2 h_{ij}^{} ) - H h_{j}^{m} h_{i}^{n} g_{mn }^{}\\
				      &= \Delta h_{ij}^{} - 2 H h_{li }^{} g_{}^{lm } h_{mj}^{} + \left| A \right| ^2 h_{ij}^{}.
			      \end{split}
		      \end{equation*}

		\item Since $H=g_{}^{ij} h_{ij}^{} $, we have that
		      \begin{equation*}
			      \begin{split}
				      \frac{\partial }{\partial t} H= \frac{\partial }{\partial t} (g_{}^{ij} h_{ij}^{} )&=\frac{\partial g_{}^{ij} }{\partial t} h_{ij}^{} + g_{}^{ij} \frac{\partial h_{ij}^{} }{\partial t} \\
				      &=2H h_{}^{ij} h_{ij}^{} + g_{}^{ij} (\Delta h_{ij}^{} - 2 H h_{li }^{} g_{}^{lm } h_{mj}^{} + \left| A \right| ^2 h_{ij}^{})\\
				      &=\Delta H + \left| A \right| ^2 H.
			      \end{split}
		      \end{equation*}

		\item Combining previous results, we can deduce the following evolution equation \begin{equation*}
			      \begin{split}
				      \frac{\partial }{\partial t} h_{i}^{\ j}
				      &= \frac{\partial }{\partial t} (h_{ik}^{} g_{}^{kj} )\\
				      &= (\Delta h_{ik}^{} - 2 H h_{li }^{} g_{}^{lm } h_{mk}^{} + \left| A \right| ^2 h_{ik}^{})g_{}^{kj}+ h_{ik}^{} (2H h_{}^{kj} )\\
				      &= \Delta h_{i}^{\ j} - 2H h_{ik}^{} h_{}^{kj} + \left| A \right| ^2 h_{i}^{\ j} - 2H h_{ik}^{} h_{}^{kj}\\
				      &= \Delta h_{i}^{\ j} + \left| A \right| ^2 h_{i}^{\ j}.
			      \end{split}
		      \end{equation*}
		      Since $\left| A \right| ^2=h_{}^{ij} h_{ij}^{} = h_{i}^{\ j} h_{\ j}^{i}$, we have that
		      \begin{equation*}
			      \begin{split}
				      \frac{\partial }{\partial t} \left| A \right| ^2
				      &= \frac{\partial }{\partial t}  (h_{i}^{\ j} h_{\ j}^{i}) \\
				      &= (\Delta h_{i}^{\ j} + \left| A \right| ^2 h_{i}^{\ j})h_{\ j}^{i} + h_{i}^{\ j}(\Delta h_{\ j}^{i} + \left| A \right| ^2 h_{\ j}^{i})\\
				      &= 2(h_{}^{ij} \Delta h_{ij}^{} + \left| A \right| ^4).\\
			      \end{split}
		      \end{equation*}
		      Since the connection $\nabla $ is compatible with the metric $g$, we have that
		      \begin{equation*}
			      \begin{split}
				      \Delta \left| A \right| ^2
				      &= g_{}^{mn } \nabla_{m}\nabla_{n} (h_{}^{ij} h_{ij}^{})  \\
				      &= 2g_{}^{mn } \nabla_{m} (h_{}^{ij} \nabla_{n}h_{ij}^{})  \\
				      &= 2(g_{}^{mn } \nabla_{m}\nabla_{n}h_{ij}^{}) h_{}^{ij} + 2g_{}^{mn } (\nabla_{m} h_{}^{ij}) (\nabla_{n}h_{ij}^{}) \\
				      &= 2 h_{}^{ij} \Delta h_{ij}^{} + 2 \left| \nabla A \right| ^2.
			      \end{split}
		      \end{equation*}
		      It follows that
		      \begin{equation*}
			      \begin{split}
				      \frac{\partial }{\partial t} \left| A \right| ^2
				      &= 2(h_{}^{ij} \Delta h_{ij}^{} + \left| A \right| ^4)\\
				      &= \Delta \left| A \right| ^2 - 2 \left| \nabla A \right| ^2 + \left| A \right| ^4.
			      \end{split}
		      \end{equation*}
	\end{enumerate}
\end{proof}
\section{Preservation of convexity and the pinching condition}
Combining the maximum principles and the evolution equations, we can prove the following two theorems in~\cite[Section 4]{huisken_flow_1984}.
\begin{theorem}
	If $h_{ij}^{} \geq 0 $ at $t=0,$ then it remains so for $0 \leq t < T.$ 
\end{theorem}
\begin{proof}
	We have that \[\frac{\partial }{\partial t} h_{ij}^{} = \Delta h_{ij}^{} - 2 H h_{li }^{} g_{}^{lm } h_{mj}^{} + \left| A \right| ^2 h_{ij}^{}.\]
	Let $M_{ij }^{} = h_{ij }^{} $ and $N_{ij }^{} = \left| A \right| ^2 h_{ij }^{} - 2 H h_{li }^{} g_{}^{lm } h_{mj}^{}.$
	If a vector $X^j$ satisfies that $h_{ij }^{} X^j=0$ for all $i$, then
	\[N_{ij }^{} X^j = \left| A \right| ^2 (h_{ij }^{} X^j) - 2 H h_{li }^{} g_{}^{lm } (h_{mj}^{}X^j)=0.\]
	Hence we can apply \autoref{thm:MP2T} to conclude.
\end{proof}

\begin{theorem} \label{stpin}
	If $\epsilon H g_{ij }^{} \leq h_{ij }^{} \leq \beta H g_{ij }^{} ,$ and $H \geq 0$ at $t=0,$ then it remains true for $t>0$.
\end{theorem}
\begin{proof}
	First, since $\frac{\partial }{\partial t} H=\Delta H+ \left| A \right| ^2 H$, by \autoref{thm:MPP} we have that if $H \geq 0$ at $t=0$, $H \geq 0$ for all $t \geq 0$.
	Let $M_{ij }^{} = h_{ij }^{} - \epsilon H g_{ij }^{} .$ Then
	\begin{equation*}
	\begin{split}
		\frac{\partial }{\partial t} M_{ij }^{}  
	&= \frac{\partial }{\partial t} h_{ij }^{} -\epsilon (\frac{\partial }{\partial t} H) g_{ij }^{} - \epsilon H \frac{\partial }{\partial t} g_{ij }^{}  \\
	&= \Delta h_{ij}^{} - 2 H h_{li }^{} g_{}^{lm } h_{mj}^{} + \left| A \right| ^2 h_{ij}^{}-\epsilon g_{ij }^{} (\Delta H+ \left| A \right| ^2 H)-\epsilon H(-2H h_{ij }^{} )\\
	&= \Delta M_{ij }^{} + \left| A \right| ^{2}h_{ij }^{} + 2 \epsilon H^2 h_{ij }^{} - \epsilon \left| A \right| ^2 H g_{ij }^{} - 2 H h_{li }^{} g_{}^{lm } h_{mj}^{}.
	\end{split}
	\end{equation*}
	Let $N_{ij }^{} = \left| A \right| ^{2}h_{ij }^{} + 2 \epsilon H^2 h_{ij }^{} - \epsilon \left| A \right| ^2 H g_{ij }^{} - 2 H h_{li }^{} g_{}^{lm } h_{mj}^{}$. From direct computation we have that 
	\begin{equation*}
	\begin{split}
		N_{ij }^{} 
		&= \left| A \right| ^2(h_{ij }^{} -\epsilon H g_{ij }^{} )-2H(h_{li }^{} g_{}^{lm } h_{mj}^{}-\epsilon H h_{ij }^{} )\\
		&= \left|A \right| ^2 M_{ij }^{} -2H(h_{li }^{} g_{}^{lm } h_{mj}^{}-\epsilon H h_{li }^{} g_{}^{lm} g_{mj }^{} )\\
		&= \left|A \right| ^2 M_{ij }^{} -2H h_{i}^{m} (h_{mj }^{}-\epsilon H  g_{mj }^{} )\\
		&= \left|A \right| ^2 M_{ij }^{} -2H h_{i}^{m} M_{ mj}^{}. \\
	\end{split}
	\end{equation*}  
	Then for a null vector $X^i$ of $M_{ij }^{} $, we have that
	\begin{equation*}
	\begin{split}
		N_{ij }^{} X^j 
	&= \left|A \right| ^2 (M_{ij }^{}X^j) -2H h_{i}^{m} (M_{ mj}^{} X^j)=0.\\
	\end{split}
	\end{equation*}
	Then the result follows from \autoref{thm:MP2T}.
\end{proof}

\section{Stampacchia's iteration}
One essential step for proving \autoref{thm:main1} is to show that the quantity $\left| A \right| ^2-\frac{1}{n}H^2$ becomes small compared to $H^2$.

\begin{theorem} \label{PinEs}
	There are constants $C_0<\infty $ and $\delta >0$ depending only on $M_0$ such that 
	\[\left| A \right| ^2-\frac{1}{n}H^2 \leq C_0 H^{2-\delta }\]
	for all times $t \in [0,T)$. 
\end{theorem}

The rationale behind the idea is that
\[\left| A \right| ^2-\frac{1}{n}H^2=\frac{1}{n}\sum_{i<j}^{n}(\kappa _i-\kappa _j)^2\]
measures the sum of differences between eigenvalues $\kappa _i$ of the second fundamental form $A$. 

An iteration scheme known as Stampacchia's iteration is used to reach the goal. In this section, we introduce the general idea for Stampacchia's iteration.

The principal components of Stampacchia's iteration are \autoref{SIalg}, an algebraic lemma by Stampacchia~\cite{SJL_1963-1964___3_1_0}, and \autoref{MSeu}, a version of the Sobolev inequality by Michael and Simon~\cite{michael_sobolev_1973}.
\begin{lemma}[{\cite[Lemma 4.1]{SJL_1963-1964___3_1_0}}] \label{SIalg}
	Let $f \colon [\bar{x},\infty) \to \R$ be a non-negative and non-increasing function. Suppose for $C>0, p>0$ and $\gamma >1,$ 
	\[ (y-x)^{p}f(y) \leq Cf(x)^{\gamma },\quad \forall y \geq x \geq \bar{x}.\] 
	Then $f(y)=0$ for $y \geq \bar{x} + d$ where $d^p=C f(\bar{x})^{\gamma -1}2^{\frac{p \gamma }{\gamma -1}}$ 
\end{lemma}
\begin{proof}
	Without loss of generality, we can assume that $\bar{x}=0.$ 
	Let $g=(\frac{f}{f(0)})^{\frac{1}{p}}$ and $A=(Cf(0)^{\gamma -1})^{\frac{1}{p}}.$ For $y \geq x \geq 0,$ we have that
	\begin{align*}
		(y-x)^{p}f(y) &\leq Cf(x)^{\gamma }\\
		A^p (y-x)^{p}f(y) &\leq A^p Cf(x)^{\gamma }\\
		(y-x)^p g(y)^p f(0)^{\gamma } &\leq C f(0)^{\gamma-1 }g(x)^{p \gamma } f(0)^{\gamma }\\
		(y-x)g(y) &\leq A g(x)^{\gamma }.
	\end{align*}
	Now fix $y>0$, let $x_n=y(1-\frac{1}{2^n})$. Note that $\lim_{n \to \infty} x_n=y$ and $x_0=0.$ Hence, we have that $g(x_0)=g(0)=1$ and 
	\begin{align*}
		(x_{n+1}-x_n)g(x_{n+1}) &\leq Ag(x_{n}^{} )^{\gamma }\\
		y(\frac{1}{2^{n}}-\frac{1}{2^{n+1}})g(x_{n+1}^{} ) &\leq Ag(x_{n}^{} )^{\gamma }\\
		g(x_{n+1}^{} ) &\leq \frac{A}{y}2^{n+1} g(x_{n}^{} )^{\gamma }.
	\end{align*} 
	Using the above inequality inductively, we have that \[g(x_{n}^{} ) \leq (\frac{A}{y})^{1+\gamma + \dots + \gamma ^{n-1}} 2^{n+(n-1)\gamma + (n-2)\gamma ^2 + \dots + \gamma ^{n-1}}.\]
	Since \[n+(n-1)\gamma + (n-2)\gamma ^2 + \dots + \gamma ^{n-1}=\frac{\gamma ^n+n-(n+1)\gamma }{(\gamma -1)^2},\]
	if we choose $y$ such that $\frac{A}{y}=2^{-\frac{\gamma }{\gamma -1}}$, then we have that 
	\begin{equation*}
	\begin{split}
		g(x_{n}^{} ) 
	&\leq  (\frac{A}{y})^{\frac{\gamma ^n-1}{\gamma -1}} 2^{\frac{\gamma ^n+n-(n+1)\gamma }{(\gamma -1)^2}} \\
	&\leq 2^{\frac{1}{(\gamma -1)^2}(-\gamma (\gamma ^n-1)+\gamma ^{n+1}+n-(n+1)\gamma )}\\
	&=2^{-\frac{n}{\gamma -1}}.
	\end{split}
	\end{equation*}
	It follows that $\lim_{n \to \infty} g(x_n)=0$. By continuity of $g$, we have that $g(y)=0.$ Therefore, $f(y)=0.$  
\end{proof}

\begin{lemma}[{\cite[Theorem 2.1]{michael_sobolev_1973}}] \label{MSeu}
    Let $v$ be a Lipschitz function on $M$. Then
    \[\left( \int_{M}^{}\left| v \right| ^{\frac{n}{n-1}} d \mu  \right)^{\frac{n-1}{n}} \leq c(n)\int_{M}^{} \left| \nabla v \right| + H \left| v \right|  d \mu. \]
\end{lemma}

The geometric quantity we aim to bound is 
\[f_\sigma = \left( \left| A \right| ^2-\frac{1}{n}H^2 \right) H^{\sigma-2} = \left( \frac{\left| A \right| ^2}{H^2}-\frac{1}{n} \right) H^{\sigma} \]
for sufficient small $\sigma >0$.

Since $M$ is uniformly convex, by \autoref{stpin}, we have that $\epsilon H g_{ij }^{} \leq h_{ij }^{} \leq \beta H g_{ij }^{} ,$ and $H \geq 0$ for any $t>0$.  Combining previous evolution equations, we can deduce as in~\cite[Corollary 5.3]{huisken_flow_1984} that
\begin{equation} \label{evof}
    \frac{\partial }{\partial t} f_\sigma \leq \Delta f_\sigma + \frac{2(1-\sigma )}{H}\left\langle \nabla _i H, \nabla_{i}^{} f_\sigma  \right\rangle - \epsilon ^2 \frac{1}{H^{2-\sigma }}\left| \nabla H \right| ^2+\sigma \left| A \right| ^2 f_\sigma
\end{equation}
for all $0 \leq t < T$ and $\sigma >0$. 

Applying integration by parts and Peter-Paul inequality, we have the following Poincare-like inequality for $f_\sigma $.

\begin{lemma}[{\cite[Lemma 5.4]{huisken_flow_1984}}]
    Let $p \geq 2$. For any $0 < \sigma \leq \frac{1}{2}$ and any $\eta >0$, we have that 
    \begin{equation*}
    \begin{split}
        n \epsilon ^2 \int_{}^{}f_{\sigma }^{p} H^2d \mu \leq& \left( 2 \eta p+5 \right) \int_{}^{}\frac{1}{H^{2-\sigma }}\left| \nabla H \right| ^2 d \mu  \\
    &+ \eta ^{-1}\left( p-1 \right) \int_{}^{}f_{\sigma }^{p-2} \left| f_{\sigma }^{}  \right| ^2 d \mu .  \\
    \end{split}
    \end{equation*} 
\end{lemma}

For a positive constant $k$, we let $f_{\sigma ,k}^{} =(f_{\sigma }^{} -k)_+$, $A(k)=\left\{ f_{\sigma }^{} \geq k \right\} $ and $A(k,t)=A(k)\cap M_t.$ Following Huisken's idea in~\cite[Lemma 5.5]{huisken_flow_1984},we can further derive a evolution-like inequality for $f_{\sigma ,k}$.

\begin{lemma}
    Let $p \geq 2$. For any $0<\sigma <1 $, we have that 
    \begin{equation*}
        \begin{split}
            \frac{\partial }{\partial t}\int_{}^{}f_{\sigma,k }^{p} d \mu 
        \leq&  -\frac{1}{2} p(p-1) \int_{}^{} f_{\sigma ,k}^{p-2} \left| \nabla f_{\sigma }^{}  \right| ^2 d \mu\\
        & -p\left( \epsilon ^2-\frac{2}{p-1} \right)\int_{}^{} f_{\sigma ,k}^{p-1} \frac{\left| \nabla H \right| ^2}{H^{2-\sigma }} d \mu  \\
        & - \int_{}^{}H^2 f_{\sigma,k }^{p} d \mu +\sigma p \int_{A(k,t)}^{} H ^2  f_{\sigma }^{p} d \mu .
        \end{split}
        \end{equation*} 
\end{lemma}

\begin{proof}
    The idea is to multiply both sides of \autoref{evof} by $p f_{\sigma ,k}^{p-1} $ and integrate by parts over $M_t$.
    For the left-hand side, we have that 
    \begin{equation*}
    \begin{split}
        \int_{}^{} p f_{\sigma ,k}^{p-1} \frac{\partial }{\partial t} f_{\sigma }^{} d \mu  
    &= \int_{}^{}\frac{\partial }{\partial t} f_{\sigma ,k}^{p} d \mu  \\
    &= \frac{\partial }{\partial t} \int_{}^{} f_{\sigma ,k}^{p} d \mu - \int_{}^{} f_{\sigma ,k}^{p} \frac{\partial }{\partial t} (d \mu) \\
    &= \frac{\partial }{\partial t} \int_{}^{} f_{\sigma ,k}^{p} d \mu + \int_{}^{}H^2 f_{\sigma ,k}^{p} d \mu . 
    \end{split}
    \end{equation*}
    For the right-hand side, 
    \begin{equation*}
    \begin{split}
        \int_{}^{} p f_{\sigma ,k}^{p-1} \Delta f_{\sigma }^{} d \mu  
    &=  -p(p-1)\int_{}^{}f_{\sigma ,k}^{p-2} \left| \nabla f_{\sigma }^{}  \right| ^2\\
    \end{split}
    \end{equation*} 
    and $ \left| A \right| ^2 \leq H^2, \left\langle \nabla _i H, \nabla_{i}^{} f_\sigma  \right\rangle \leq \left| \nabla H \right| \left| \nabla f_{\sigma }^{}  \right| .$ 
    It follows that 
    \[f_{\sigma ,k}^{} \leq f_{\sigma }^{} = \left( \left| A \right| ^2-\frac{1}{n}H^2 \right) H^{\sigma-2} \leq H^{\sigma } \]
    and for $0<\sigma <1, p \geq 2$ 
    \begin{equation*}
    \begin{split}
        \frac{2(1-\sigma )}{H}f_{\sigma,k }^{} \left| \nabla H \right| \left| \nabla f_{\sigma }^{}  \right| 
    &\leq \frac{p-1}{2} \left| \nabla f_{\sigma }^{}  \right|^2 + \frac{2}{p-1} \frac{\left| \nabla H \right| ^2 f_{\sigma ,k}^{2} }{H^2}  \\
    & \leq \frac{p-1}{2} \left| \nabla f_{\sigma }^{}  \right|^2 + \frac{2}{p-1} \frac{\left| \nabla H \right| ^2 }{H^{2-\sigma }} f_{\sigma ,k}^{}   \\
    \end{split}
    \end{equation*} 
    Hence
    \begin{equation*}
        \begin{split}
            \frac{\partial }{\partial t}\int_{}^{}f_{\sigma,k }^{p} d \mu 
            &+ p(p-1)\int_{}^{}f_{\sigma,k }^{p-2} \left| \nabla f_\sigma  \right| ^2 d \mu \\
            &+ \epsilon ^2 p \int_{}^{}\frac{1}{H^{2-\sigma }}f_{\sigma,k }^{p-1} \left| \nabla H \right| ^2 d \mu + \int_{}^{}H^2 f_{\sigma,k }^{p} d \mu  \\ 
        \leq & 2(1-\sigma)p \int_{}^{} \frac{1}{H}f_{\sigma,k }^{p-1} \left| \nabla H \right| \left| \nabla f_{\sigma }^{}  \right| d \mu + \sigma p \int_{}^{} \left| A \right| ^2 f_{\sigma,k }^{p-1} f_{\sigma }^{} d \mu .  \\
        \leq & \frac{1}{2} p(p-1) \int_{}^{} f_{\sigma ,k}^{p-2} \left| \nabla f_{\sigma }^{}  \right| ^2 d \mu + 2 \frac{p}{p-1} \int_{}^{} f_{\sigma ,k}^{p-1} \frac{\left| \nabla H \right| ^2}{H^{2-\sigma }} \\
        &+ \sigma p \int_{A(k,t)}^{} H ^2  f_{\sigma }^{p} d \mu .  \\
        \end{split}
        \end{equation*}
    Therefore,
    \begin{equation*}
    \begin{split}
        \frac{\partial }{\partial t}\int_{}^{}f_{\sigma,k }^{p} d \mu 
    \leq&  -\frac{1}{2} p(p-1) \int_{}^{} f_{\sigma ,k}^{p-2} \left| \nabla f_{\sigma }^{}  \right| ^2 d \mu\\
    & -p\left( \epsilon ^2-\frac{2}{p-1} \right)\int_{}^{} f_{\sigma ,k}^{p-1} \frac{\left| \nabla H \right| ^2}{H^{2-\sigma }} d \mu  \\
    & - \int_{}^{}H^2 f_{\sigma,k }^{p} d \mu +\sigma p \int_{A(k,t)}^{} H ^2  f_{\sigma }^{p} d \mu .
    \end{split}
    \end{equation*} 
\end{proof}

Now we have established two inequalities for the function $f_{\sigma }^{} $. Notice that any compact hypersurface $M$ in $\R^{n+1}$ can be enclosed by a sphere which shrinks to a point under MCF in finite time. From the avoidance principle, we have that the maximal time $T< \infty $. Then by \autoref{stamit}, a general iteration scheme we are going to derive in the later chapter, we can bound $f_{\sigma }^{} $ uniformly for all times $t \in [0,T)$, which proves \autoref{PinEs}. 

\section{Convergence to a round point} \label{sec:crp}

A general argument from Huisken \cite[Theorem 8.1]{huisken_flow_1984} states that $M_t$ exists on a maximal time interval $t \in [0,T)$ where $T<\infty $ and $\max_{M_t}\left| A \right| ^2 $ becomes unbounded as $t \to T$.

First, we show that $M_t$ converges to a single point. To control the diameter of $M_t$, we need to examine the minimum value of the mean curvature $H_{\min }$. Since $\left| A \right| ^2 \leq H^2$, the maximum value of the mean curvature $H_{\max } \to \infty$ as $t$ approaches $T$. To compare the mean curvature at different points on $M_t$, we need the following gradient estimate for $H$.
\begin{theorem}[{\cite[Theorem 6.1]{huisken_flow_1984}}] \label{graEs}
	For any $\eta >0$, there exists a constant $C=C(\eta ,M_0,n)$ such that 
	\[\left| \nabla H \right| ^2 \leq \eta H^4 + C.\]
\end{theorem}
Following Huisken's treatment in the proof of~\cite[Theorem 8.4]{huisken_flow_1984}, combining $\autoref{graEs}$ and the preservation of curvature pinching $h_{ij }^{} \leq \epsilon H g_{ij }^{} $ and Myer's theorem one can prove that \[\frac{H_{\max }}{H_{\min }} \to 1\] as $t \to T$. Hence the diameter of $M_t$ decreases to zero as $t \to T.$ 

As for the normalized hypersurface $\tilde{M}_t$ parametrized by $\tilde{F}(\cdot ,t)=\psi (t) \cdot F(\cdot ,t)$, geometric quantities of $\tilde{M}_t$ are rescaled as follows:
\[\tilde{g }_{ij }^{}=\psi ^2 g_{ij }^{}, \quad  \tilde{h}_{ij}=\psi h_{ij }^{} , \quad  \tilde{H}=\psi ^{-1} H. \]
Hence
\begin{equation*}
\begin{split}
	\frac{\partial }{\partial t} \sqrt[]{\det \tilde{g}_{ij }^{} } 
&=  \frac{\det \tilde{g}_{ij }^{}}{2 \sqrt[]{\det \tilde{g}_{ij }^{}}} \tilde{g}_{ }^{pq} \frac{\partial \tilde{g}_{pq }^{} }{\partial t}  \\
&= \frac{\sqrt[]{\det \tilde{g}_{ij }^{} }}{2}\psi ^{-2}g_{ }^{pq} \left( \frac{\partial \psi^2 }{\partial t} g_{ pq}^{}  + \psi^2 \frac{\partial g_{ pq}^{} }{\partial t}  \right) \\
&= \sqrt[]{\det \tilde{g}_{ij }^{} }\left(\psi ^{-1}n  \frac{\partial \psi }{\partial t}   - \psi ^2 \tilde{H}^2  \right)
\end{split}
\end{equation*}
Therefore, differentiating \autoref{fixarea} yields that
\[\psi ^{-1}  \frac{\partial \psi }{\partial t}   = \frac{1}{n}\psi ^2 \tilde{h}.\]
Then for $\tilde{t}(t)=\int_{0}^{t}\psi ^2(\tau )d \tau $, we have the following normalized flow equation on a different maximal time interval $\tilde{t} \in [0,\tilde{T})$:
\begin{equation*}
	\frac{\partial \tilde{F}}{\partial \tilde{t}}=-\tilde{H}\tilde{\nu }+ \frac{1}{n}\tilde{h}\tilde{F}
\end{equation*}

The evolution equations of geometric quantities under the normalized flow differ from the original evolution equations by a lower order term. Most of the computations in previous sections still hold. We can further prove that the maximal time $\tilde{T}=\infty $ and derive the exponential decay of the following geometric quantities on $\tilde{M}_t$.
\begin{lemma} [{\cite[Lemma 10.6]{huisken_flow_1984}}] \label{exdc}
	There are constants $\delta >0$, $C~<~\infty $ depending only on geometric quantities of $M_0$ and the dimension $n$ such that
	\begin{enumerate}[\normalfont(1)]
		\item $| \tilde{A} | ^2-\frac{1}{n}\tilde{H}^2 \leq Ce^{-\delta \tilde{t}}$ 
		\item $\left| \tilde{h}_{ij} \tilde{H} - \frac{1}{n}\tilde{h}\tilde{g}_{ij }^{}  \right|\leq Ce^{-\delta \tilde{t}}$ 
		\item $\max_{\tilde{M}}\left| \nabla_{}^{m} \tilde{A} \right| \leq C_m e^{-\delta _m \tilde{t}}$  
	\end{enumerate}
	where $\delta_m>0, C_m< \infty $ also depend on the order $m$.
\end{lemma} 
Since the metric $\tilde{g}_{ij }^{} $ evolves under the equation \[\frac{\partial }{\partial \tilde{t}} \tilde{g}_{ij }^{} = -2\tilde{h}_{ij} \tilde{H} +\frac{2}{n}\tilde{h}\tilde{g}_{ij }^{},\]
by $\autoref{exdc}$(2), $\tilde{g}_{ij }^{}$ converges uniformly to a positive definite metric $\tilde{g}_{ij }^{} (\infty )$ as $\tilde{t} \to \infty $. Then from $\autoref{exdc}$(3) and Arzela-Ascoli thoerem, we have that $\tilde{g}_{ij }^{} (\infty )$ is smooth. Finally $\autoref{exdc}$(1) implies that $\tilde{g}_{ij }^{} (\infty )$ is the metric of a sphere.
\section{Generalizations}

Huisken's convergence theorem was generalized in various settings.

In 1986, Huisken~\cite{huisken_contracting_1986} managed to extend \autoref{thm:main1} for hypersurfaces in a general Riemannian manifold where hypersurfaces need to be convex enough to overcome the obstruction caused by the curvature of the ambient manifold. WHen the ambient manifold is a spherical space-form, Huisken~\cite{huisken_deforming_1987} observed that the hypersurface could converge to a round point without the initial convexity condition. Similarly, the result obtained by Andrews and Baker~\cite{andrews_mean_2010} for the convergence of higher-codimension submanifold also allows some non-convex condition to be preserved.

For MCF with free boundary, Stahl~\cite{stahl_convergence_1996,stahl_regularity_1996} showed that if the barrier surface in the Euclidean space is a flat hyperplane or a round hypersphere, any convex hypersurface with free boundary on the barrier will converge to a round half-point. Later in 2020, Hirsch and Li~\cite{hirsch2020contracting} managed to generalize the above result to non-umbilic barriers in $\R^3$. They proved that if the barrier surface satisfies a uniform bound on the exterior and interior ball curvature and certain bounds on the first and second derivative of the second fundamental form, then sufficiently convex hypersurfaces with free boundary will converge to a round half-point.

In summary, most convergence results above were obtained by following Huisken's line of argument. The key steps are \autoref{PinEs}, the pinching estimate of the traceless second fundamental form, and \autoref{graEs}, the gradient estimate of the mean curvature. The former describes the ``roundness" of the hypersurface pointwisely, while the latter enables us to compare mean curvatures of the hypersurface at different points. In particular, the gradient estimate is built upon the pinching estimate. To prove a general convergence theorem for MCF with free boundary in the Riemannian ambient space, we need to establish a proper iteration scheme to prove the pinching estimate.




\chapterend

