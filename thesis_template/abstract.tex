
The mean curvature flow, first formulated by Mullins~\cite{mullins_twodimensional_1956}, describes the evolution of surfaces moving along its mean curvature vector. Among versions of mean curvature flow for surfaces with boundary, the mean curvature flow with free boundary is one of the most extensively studied formulations. The flow is of great importance and relates closely to free boundary minimal surfaces and constrained motion of liquid interfaces.

This thesis aims at providing a theoretical foundation for the convergence theory of the mean curvature flow with free boundary in the Riemannian ambient manifold. By reviewing the classical method by Huisken\cite{huisken_flow_1984}, we highlight the importance of the iteration scheme for proving the pinching estimate of the traceless second fundamental form. Next, we calculate the boundary derivative of the second fundamental form under the covariant formulation of the mean curvature flow introduced by Andrews and Baker~\cite{andrews_mean_2010}. We also generalize~\cite[Theorem 3.1]{edelen_convexity_2016}, an iteration scheme for proving uniform bound of functions satisfying two special inequalities, to Riemannian manifolds as the ambient space. The results of this thesis are expected to help generalize the convergence result by Hirsch and Li~\cite{hirsch2020contracting} to free boundary mean curvature flow in Riemannian manifolds.


