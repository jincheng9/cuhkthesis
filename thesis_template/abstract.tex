
Mean curvature flow, first formulated by Mullins~\cite{mullins_twodimensional_1956}, describes the evolution of surfaces moving along its mean curvature vector. Among various boundary conditions for mean curvature flow with boundary, the free boundary condition is one of the most extensively studied formulations. Mean curvature flow with free boundary is of great significance and relates closely to free boundary minimal surfaces and constrained motion of liquid interfaces.

This thesis provides a theoretical foundation for the convergence theory of mean curvature flow with free boundary in Riemannian ambient manifolds. By reviewing the classical method by Huisken\cite{huisken_flow_1984}, we highlight the importance of an iteration scheme for proving the pinching estimate of the traceless second fundamental form. Next, we calculate the boundary derivative of the second fundamental form under the covariant formulation of mean curvature flow introduced by Andrews and Baker~\cite{andrews_mean_2010}. We also generalize~\cite[Theorem 3.1]{edelen_convexity_2016}, an iteration scheme for proving uniform bound of functions satisfying two special inequalities, to Riemannian manifolds as the ambient space. The results of this thesis will help generalize the convergence result by Hirsch and Li~\cite{hirsch2020contracting} to mean curvature flow with free boundary in Riemannian manifolds.


