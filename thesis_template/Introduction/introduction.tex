%
% chapter introduction

\chapter{Introduction}

\section{Background}

The last few decades have witnessed significant development in the field of geometric flow, which leads to many remarkable accomplishments in geometry, topology, physics, and computer vision. Among various geometric flows, mean curvature flow (MCF) is one of the most important geometric flows for submanifolds of Riemannian manifolds. One way of understanding MCF is to regard it as the negative gradient flow for the area functional. In other words, a surface is deforming along MCF to decrease its area as fast as possible.

The study of MCF and other related flows is a crucial area of mathematics. Not only does it lead to a series of significant results in physics and mathematics, but it is also expected to solve some long-standing conjectures in geometry and topology. In 1994, Andrews~\cite{andrews_contraction_1994} applied the harmonic MCF to provide a new proof for the topological sphere theorem. Moreover, regarded as possible evidence to the cosmic censorship conjecture in General Relativity, the Riemannian Penrose inequality was proved by Huisken and Ilmanen~\cite{huisken_inverse_2001} using the method of inverse MCF. For further applications, inspired by similarities between Ricci flow and MCF and the resolution of Thurston's Geometrization Conjecture by Perelman using the Ricci flow, mathematicians believe that MCF could be a possible approach to the Schoenflies Conjecture in geometric topology~\cite{chodosh2020mean,bernstein2020closed}.

The nonlinear nature of geometric flows leads to the possible appearance of singularities. Mathematicians have developed various methods to continue the flow through singularities. In the framework of differential geometry, by analogy with Hamilton~\cite{Hamilton1997FourmanifoldsWP} and Perelman's~\cite{perelman2002entropy,perelman2003ricci} construction of Ricci flow with surgery, mathematicians attempted to perform surgeries before the formation of singularities to continue the flow. Huisken and Sinestrari~\cite{Huisken2009MeanCF} applied this idea successfully to $2$-convex hypersurfaces under MCF in $\R^n(n \geq 4)$. A few years later, Brendle and Huisken~\cite{brendle_mean_2016} managed to extend the result to mean convex surfaces in $\R^3$. Although MCF with surgery provides better control on the topology and stays inside the smooth category, it requires technical virtuosity and deep understandings of how singularities are formed.

Another way of continuing beyond the singular time is to consider a class of generalized solutions or weak solutions allowing singularities. Using tools from geometric measure theory, Almgren~\cite{almgren_plateaus_1966} and Allard~\cite{allard_first_1972} introduced and developed theories on a generalized class of surfaces called varifolds. Brakke~\cite{Brakke_1978} later defined the MCF equation in the space of varifolds by certain transport inequalities and proved a general existence theorem for the flow. The flow is referred to as the Brakke flow.

In material science, MCF arises naturally in describing the evolution of interfaces that bound phases of materials. Mullins~\cite{mullins_twodimensional_1956} first formulated the MCF equation to model grain boundaries during metal annealing. Before the 1990s, most results on MCF were established for hypersurfaces without boundary. However, although being considerably more challenging than the no boundary case, the study of MCF for surfaces with boundary is of great significance. It is a more natural way to describe physical phenomena. For instance, the deformation of grain boundaries or the evolution of interfaces usually happens in some containers under certain physical boundary conditions. MCF with boundary can best describe such scenarios. Applications of MCF with boundary also include describing the motion of soap film whose boundary moves freely in a fixed surface.

To define MCF for surfaces with boundary properly, mathematicians mainly focus on two geometric boundary conditions, the Dirichlet boundary condition, and the Neumann boundary condition. For the Dirichlet boundary condition where the boundary moves in a prescribed way, Huisken~\cite{huisken_non-parametric_1989} proved a theorem for graphs under the non-parametric mean curvature flow using the classical theory of Lieberman~\cite{Lieberman_1986} regarding general quasilinear parabolic equations with Dirichlet boundary conditions. Generalizations to the Riemannian settings have been introduced by Priwitzer~\cite{priwitzer_mean_2003}, and weak formulations for fixed boundary conditions have appeared in several studies~\cite{stuvard_existence_2021,white2021mean}.

For the Neumann boundary condition, the angle between the evolving surface and the barrier is prescribed. When the contact angle is fixed to be $\pi/2$, the flow is called MCF with free boundary.Huisken~\cite{huisken_non-parametric_1989} studied MCF with free boundary for graphs using the non-parametric method and proved the long-time existence of the solutions and the convergence of the solutions to a plane domain. After a few years, Stahl~\cite{stahl_regularity_1996} established the fundamental existence and uniqueness theorem for MCF with free boundary in the parametric setting. As for boundary singularities, Buckland~\cite{buckland_mean_2005} proved a boundary monotonicity formula to classify certain boundary singularities of MCF with free boundary. Volkmann~\cite{volkmann_monotonicity_2016} also proved a monotonicity formula for compact free boundary surfaces with square-integrable mean curvature in the unit ball, which leads to a Li--Yau type inequality.

More generally, for MCF with arbitrary fixed contact angle, Altschuler and Wang~\cite{altschuler_translating_1994} proved the long-time existence of the flow for graphs in $\R^2$ and showed that the graphs would converge to translating solutions. The existence and convergence theorems were extended by Guan~\cite{guan1996mean} for higher-dimensional graphs. Furthermore, Bellettini and Kholmatov~\cite{bellettini_minimizing_2018} are concerned about the case of possibly nonconstant prescribed contact angle to describe the motion of droplets flowing on hyperplanes.
 
In 1984, Huisken~\cite{huisken_flow_1984} published his seminal paper and proved that closed convex hypersurfaces in Euclidean spaces of dimension at least three would converge under MCF to a round point in finite time. Subsequently, Stahl~\cite{stahl_convergence_1996} generalized Huisken's result for MCF with free boundary where the barrier is umbilic. In 2020, Hirsch and Li~\cite{hirsch2020contracting} proved a convergence theorem of MCF with free boundary on non-umbilic barriers in $\R^3$. It is natural to ask whether a similar convergence result holds when the barrier is defined in a general Riemannian manifold and higher dimensions.

\section{Structure of the thesis}

The purpose of this thesis is to establish necessary theoretical tools for proving convergence theorems of MCF with free boundary in Riemannian manifolds. In \autoref{chap:crmcf}, we first review some classical results for MCF in the Euclidean space to introduce the essential ingredients for the convergence theory of MCF. Then we briefly introduce the generalization of the convergence results in the free boundary setting and discuss the similarities and differences between classical MCF and MCF with free boundary. In \autoref{chap:fbmcf}, we compute the boundary derivative of the second fundamental form and prove an iteration scheme for MCF with free boundary in a general Riemannian manifold.

\chapterend