%
% chapter introduction

\chapter{Introduction}

\note{Summary}
{
The introduction comes here.
}

The last few decades have witnessed a significant development in the field of geometric flow, which leads to many remarkable accomplishments in geometry, topology, physics and computer vision. Among various geometric flows, the Mean Curvature Flow is one of the most important geometric flows for submanifolds of Riemannian manifolds. One way of understanding the Mean Curvature Flow is to regard it as the negative gradient flow for area. In other words, a surface is deforming along the Mean Curvature Flow to decrease its area as fast as possible.

The study of mean curvature flow and the related field is a critical area of mathematics. Not only does it lead to a series of significant results in physics and mathematics, but it is also expected to solve some long-standing conjectures in geometry and topology. In 1994, Andrews [And94] applied the harmonic mean curvature flow to provide a new proof for the topological sphere theorem and improve the result of homeomorphism to a weaker version of diffeomorphism. Moreover, regarded as possible evidence to the cosmic censorship conjecture, the Riemannian Penrose inequality in general relativity was solved by Huisken and Ilmanen [HI01] using the method of inverse mean curvature flow. For further applications, inspired by similarities between Ricci flow and mean curvature flow and the resolution of Thurston's geometrization conjecture by Perelman using the Ricci flow, mathematicians believe that the mean curvature flow could be a possible way of solving the Schoenflies Conjecture in geometric topology.

Mullins [Mul56] first formulated the mean curvature flow equation to model grain boundaries during metal annealing. Before the 1990s, most results on mean curvature flow concern hypersurfaces without boundary. However, although being considerably more challenging than the no boundary case, the study of mean curvature flow for hypersurfaces with boundaries is of great significance. It is a more natural way of describing physical phenomena. For instance, the deformation of grain boundaries usually happens in some containers which provide constraints for the evolution. Such a scenario can be best described by mean curvature flow with boundaries. Possible applications of mean curvature flow with boundaries also include describing the motion of soap film whose boundary moves freely in a fixed surface.

To define the mean curvature flow for surfaces with boundaries properly, one needs to prescribe certain geometric boundary conditions. One of the most extensively studied boundary conditions is the Neumann boundary condition where the hypersurface's boundary could move freely in a prescribed barrier surface. Moreover, the angle between the hypersurface and the barrier is fixed. When the fixed contact angle is 90 degrees, the flow is then called Mean Curvature Flow with free boundary. For simplicity, it will be referred to as MCF with free boundary in the rest of the proposal.
 
The convergence theory for MCF has been developed rapidly due to the efforts of mathematicians including Gerhard Huisken, Ben Andrews and Charles Baker. In 1984, Huisken [Hui84] published his seminal paper and proved that closed convex hypersurfaces in Euclidean spaces of dimension at least three would converge under mean curvature flow to a round point in finite time. A few years later, Huisken [Hui86] managed to extend the result for hypersurfaces in a general Riemannian manifold where the hypersurface need to be convex enough to overcome the obstruction caused by the curvature of the ambient manifold to converge to a round point.

For MCF with free boundary, Stahl [St96] showed that if the barrier surface in the Euclidean space is a flat hyperplane for a round hypersphere, any convex hypersurface with free boundary on the barrier will converge to a round half point. Later in 2020, Hirsch and Li [HL20] managed to generalize the above result to non-umbilic barriers in $R^3$. They proved that if the barrier surface satisfies a uniform bound on the exterior and interior ball curvature and certain bounds on the first and second derivative of the second fundamental form, then sufficiently convex free boundary hypersurfaces will converge to a round half point.

We want to explore the following scenario further based on the previous results. Hirsch and Li [HL20] have improved Stahl's result [St96] for a more general barrier, but the barrier is still defined in the Euclidean space. Hence, we would like to consider the case where the barrier is defined in a general 3-manifold and study the behaviour of free boundary MCF to obtain some convergence result.

To prove that a free boundary hypersurface converges to a round half-point under the MCF, the standard argument from Huisken [Hui84] also works. Hence, it suffices to prove the pinching estimate by the Stampacchia's iteration and the gradient estimate. 

However, for a general barrier in a 3-manifold, two difficulties need to be overcome. First, as the prerequisite of proving pinching and gradient estimates, the initial convexity condition is expected to be preserved along the flow. The boundary derivatives are also essential for applying maximum principles to prove the preservation of properties. The problem is that the barrier is not umbilic in the current case, where some cross terms will appear in the boundary derivatives. Such cross-terms are very hard to control; thus, the maximum principle could not be applied as usual.

The second difficulty is the reformulation of Stampacchia's iteration in a more general setting. As discussed previously, Edelen [Ed16] has introduced the free boundary version of Stampacchia's iteration, but the iteration argument only works when the barrier is in the Euclidean space. To further extend the iteration argument to the Riemannian manifold, we first need to extend the Michael-Simon inequality for Riemannian manifold by Hoffman and Spruck [HS74] to the free boundary case. Then by the arguments in [Ed16], the Stampacchia's iteration could be applied once the Poincare-like inequality and the evolution-like inequality are established.

\section{Outline of Main Results}

In this thesis, our goal is to build a theoretical fundation for the convergence theory of the free boundary MCF. We managed to prove a Stampacchia's iteration scheme for the free boundary MCF in a general Riemannian manifold and computed the boundary derivative for the second fundamental form.

\section{Structure of the Thesis}

In this thesis, we first review some classical results for the MCF of convex hypersurfaces in the Euclidean space to introduce the essential ingredients for the convergence theory of the MCF. Then we briefly introduce the generalization of the convergence results in the free boundary setting and discuss the similarities and differences between the classical MCF and the MCF with free boundary. Finally. we prove the main results. 

\chapterend