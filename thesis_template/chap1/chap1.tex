%
% chapter introduction

\chapter{Introduction}

\section{Background}

The last few decades have witnessed a significant development in the field of geometric flow, which leads to many remarkable accomplishments in geometry, topology, physics and computer vision. Among various geometric flows, the Mean Curvature Flow is one of the most important geometric flows for submanifolds of Riemannian manifolds. One way of understanding the Mean Curvature Flow is to regard it as the negative gradient flow for area. In other words, a surface is deforming along the Mean Curvature Flow to decrease its area as fast as possible.

The study of mean curvature flow and the related field is a critical area of mathematics. Not only does it lead to a series of significant results in physics and mathematics, but it is also expected to solve some long-standing conjectures in geometry and topology. In 1994, Andrews \cite{andrews_contraction_1994} applied the harmonic mean curvature flow to provide a new proof for the topological sphere theorem and improve the result of homeomorphism to a weaker version of diffeomorphism. Moreover, regarded as possible evidence to the cosmic censorship conjecture, the Riemannian Penrose inequality in general relativity was solved by Huisken and Ilmanen \cite{huisken_inverse_2001} using the method of inverse mean curvature flow.

Mullins \cite{mullins_twodimensional_1956} first formulated the mean curvature flow equation to model grain boundaries during metal annealing. Before the 1990s, most results on mean curvature flow concern hypersurfaces without boundary. However, although being considerably more challenging than the no boundary case, the study of mean curvature flow for hypersurfaces with boundaries is of great significance. It is a more natural way of describing physical phenomena. For instance, the deformation of grain boundaries usually happens in some containers which provide constraints for the evolution. Such a scenario can be best described by mean curvature flow with boundaries. Applications of mean curvature flow with boundaries also include describing the motion of soap film whose boundary moves freely in a fixed surface.

To define the mean curvature flow for surfaces with boundaries properly, one needs to prescribe certain geometric boundary conditions. One of the most extensively studied boundary conditions is the Neumann boundary condition where the surface's boundary could move freely in a prescribed barrier surface. Moreover, the angle between the surface and the barrier is fixed. When the fixed contact angle is $90$ degrees, the flow is then called Mean Curvature Flow with free boundary. For simplicity, it will be referred to as MCF with free boundary in the rest of the thesis.
 
In 1984, Huisken \cite{huisken_flow_1984} published his seminal paper and proved that closed convex hypersurfaces in Euclidean spaces of dimension at least three would converge under mean curvature flow to a round point in finite time. Subsequently, Stahl \cite{stahl_convergence_1996} generalized Huisken's result for free boundary MCF where the barrier is umbilic. In 2020, Hirsch and Li \cite{hirsch2020contracting} proved the convergence thoerem of free boundary MCF with non-umbilic barriers in $\R^3$. It is natural to ask whether a similar convergence result holds when the barrier is defined in a general Riemannian manifold.

\section{Structure of the Thesis}

This thesis aims at building a theoretical foundation for the convergence theory of the MCF with free boundary in a general Riemannian manifold.

We first review some classical results for the MCF of convex hypersurfaces in the Euclidean space to introduce the essential ingredients for the convergence theory of the MCF. Then we briefly introduce the generalization of the convergence results in the free boundary setting and discuss the similarities and differences between the classical MCF and the MCF with free boundary. Finally. we compute the boundary derivative for the second fundamental form and prove a Stampacchia’s iteration scheme for the MCF with free boundary in a general Riemannian manifold.

\chapterend