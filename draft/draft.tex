\documentclass[a4paper]{report}

\usepackage[english]{babel}
\usepackage[utf8]{inputenc}
\usepackage{amsmath}
\usepackage{amssymb}
\usepackage{amsthm}
\usepackage{graphicx}
\usepackage[colorinlistoftodos]{todonotes}
\usepackage[colorlinks,
linkcolor=blue,
anchorcolor=blue,
citecolor=green
]{hyperref}
\usepackage{tikz-cd}
%\usepackage{csquotes}
%\usepackage[hyphens,spaces,obeyspaces]{url}


\newtheorem{theorem}{Theorem}
\newtheorem{lemma}{Lemma}
\newtheorem{cor}{Corollary}[theorem]
\newtheorem{prop}[theorem]{Proposition}
\newtheorem{conj}{Conjecture}
\newtheorem{defn}{Definition}

\newcommand{\bigM}{\mathcal{M}}
\newcommand{\N}{\mathbb{N}}
\newcommand{\Z}{\mathbb{Z}}
\newcommand{\Q}{\mathbb{Q}}
\newcommand{\R}{\mathbb{R}}
\newcommand{\hp}{\mathbb{H}}
\newcommand{\C}{\mathbb{C}}
\newcommand{\bbu}{\mathbb{U}}

\newcommand{\bigo}{\mathcal{O}}

\newcommand{\E}{\mathcal{E}}
\newcommand{\F}{\mathcal{F}}
\newcommand{\G}{\mathcal{G}}
\newcommand{\U}{\mathcal{U}}
\newcommand{\V}{\mathcal{V}}
\newcommand{\calH}{\mathcal{H}}
\newcommand{\calQ}{\mathcal{Q}}

\newcommand{\exprm}{\mathrm{exp}}
\newcommand{\resrm}{\mathrm{Res}}

\newcommand{\xto}{\xrightarrow}

\theoremstyle{remark}
\newtheorem*{rem}{Remark}
\newtheorem*{note}{Note}

\title{Convergence Theories of the Mean Curvature Flow}

\author{Yizi WANG}

\date{\today}

\begin{document}
\maketitle

\section{Maximum Principles and Preliminary Geometric Identities}

\begin{lemma}
	[Simon's identity]
	\[\Delta h_{ij}^{} = \nabla_{i}\nabla_{j} H + H h_{li}^{} g_{}^{lm } h_{mj}^{} - \left| A \right| ^2 h_{ij}^{} \]
\end{lemma}

\begin{proof}
	Note that $\Delta h_{ij}^{} = g_{}^{mn} \nabla_{m}\nabla_{n} h_{ij}^{} $ and $\nabla_{i}\nabla_{j} H = g_{}^{mn } \nabla_{i}\nabla_{j} h_{mn }^{} $. It suffices to examine the difference $\nabla_{m}\nabla_{n} h_{ij}^{} - \nabla_{i}\nabla_{j} h_{mn}^{} .$ Since the ambient space is Euclidean, from the Codazzi equation we have that $\nabla_{i}^{} h_{j}^{k} = \nabla_{j}^{} h_{i}^{k}$. Hence \[\nabla_{m}\nabla_{n} h_{ij}^{} - \nabla_{i}\nabla_{j} h_{mn}^{} = \nabla_{m}\nabla_{i} h_{nj}^{} - \nabla_{i}\nabla_{m} h_{jn}^{}=(\nabla_{m}\nabla_{i}  - \nabla_{i}\nabla_{m})h_{nj}^{}.\] By the product rule of connections acting on tensor product, we have that \[(\nabla_{m}\nabla_{i}  - \nabla_{i}\nabla_{m})h_{nj}^{} = R_{min}^{\quad\  l} h_{lj}^{} + R_{mij}^{\quad \ l} h_{nl }^{}.  \]
	Therefore, by Gauss equation
	\begin{equation*}
		\begin{split}
			\Delta h_{ij}^{} - \nabla_{i}\nabla_{j} H
			&= g_{}^{mn} (R_{min }^{\quad \ l} h_{lj}^{} + R_{mij}^{\quad \ l} h_{nl }^{} ) \\
			&= g_{}^{mn } g_{}^{kl} \{(h_{mn}^{} h_{ik }^{} -h_{mk }^{} h_{in }^{} )h_{lj}^{} + (h_{mj }^{} h_{ik }^{} - h_{mk }^{} h_{ij}^{} )h_{ln}^{} \}\\
			&= H g_{}^{kl} h_{ik }^{}  h_{lj}^{} - g_{}^{mn } g_{}^{kl } h_{mk }^{} h_{ln }^{} h_{ij}^{} \\
			&= H g_{}^{kl} h_{ik }^{}  h_{lj}^{} - \left| A \right| ^2 h_{ij}^{} .
		\end{split}
	\end{equation*}
\end{proof}

Parabolic maximum principles are essential PDE tools in the analysis of mean curvature flow. We will briefly introduce two frequently used versions in this section. One is the standard parabolic maximum principle for scalar functions. The other is the parabolic maximum principle for symmetric two-tensors.

\begin{theorem} \label{thm:MPP}[Strong maximum principle for parabolic equations]
	Let $M$ be a closed smooth manifold and $f: M \times [0,T) \to \R$ be a scalar function on $M$ varying along time $t$. Suppose $f (\cdot,0) \geq 0$ and
	\[
		\frac{\partial f}{\partial t} \geq \Delta f + b^{i} \nabla_{i}^{} f+ cf
	\]
	for some smooth function $b^i,c$, where $c \geq 0.$ Then
	\[\min _M f (\cdot,t) \geq \min _M f (\cdot,0).\]
	Furthermore, if there exist some $p \in M$ and $t_0 \geq 0$ such that $f(p,t_0)=\min _M f (\cdot,0) $, then $f \equiv \min _M f (\cdot,0)$ for $0 \leq t \leq t_0.$
\end{theorem}

Now we extend the maximum principle to tensors. Let $M_{ij}$ be a symmetric tensor on a closed manifold $M$. We say $M_{ij }^{} \geq 0 $ if for any vector $X$ on $M$, $M_{ij }^{} X^i X^j \geq 0$. Let $N_{ij }^{} = P(M_{ij }^{} , g_{ij}^{} )$ be another symmetric tensor formed by contracting $M_{ij }^{} $ with itself using the metric where $p$ is a polynomial. Then we have the following verison of the maximum principle:

\begin{theorem}[Strong maximum principle for symmetric two-tensors] \label{thm:MP2T}
	Suppose $M_{ij }^{} $ is a symmetric tensor on a closed manifold $M$ depending on time $t$ and on $0 \leq t < T$ satisfies that \[\frac{\partial }{\partial t} M_{ij }^{} = \Delta M_{ij }^{} + u^k \nabla_{k}^{} M_{ij }^{} + N_{ij }^{} \] where $u^k$ is a vector on $M$ and $N_{ij }^{} $ is defined as above such that 
	\[
		N_{ij }^{} X^i X^j \geq 0 \text{   whenever  } M_{ij }^{} X^j =0.
	\]
	Then if $M_{ij }^{} \geq 0$ at $t=0$, it will remain so on $0 \leq t \leq T$.
\end{theorem}

\begin{proof}
	Let $\delta >0$ be a constant depending only on $\max \left| M_{ij }^{}  \right| $.
	Set \[\tilde{M}_{ij }^{}  = M_{ij }^{} + \epsilon (\delta +t) g_{ij}^{} \] for some $\epsilon >0$. Now it suffices to show that $\tilde{M}_{ij }^{} >0$ on $0 \leq t \leq \delta $ for all $\epsilon >0$.
	Suppose for contradiction that the above assertion is not true. Then there exists $t_0 \in (0,\delta ]$ and a unit vector $X^i $ at $x_0 \in M$ such that $\tilde{M}_{ij }^{} X^j=0$ for all $i$  at $x_0$. Note that $N_{ij }^{} = P(M_{ij }^{} , g_{ij}^{} )$, we set $\tilde{N_{ij }^{} }=P(\tilde{M}_{ij }^{} , g_{ij }^{} )$. By the assumption, since $\tilde{M}_{ij }^{} X^j=0$, we have that $\tilde{N_{ij }^{}} X^i X^j \geq 0$. Then at $(x_0,t_0)$, 

	\begin{equation*}
	\begin{split}
		N_{ij }^{} X^i X^j
	&= \tilde{N_{ij }^{}} X^i X^j + (N_{ij }^{} - \tilde{N_{ij }^{}}) X^i X^j\\
	& \geq (N_{ij }^{} - \tilde{N_{ij }^{}}) X^i X^j\\
	& \geq -\left| N_{ij }^{} - \tilde{N_{ij }^{}} \right|.
	\end{split}
	\end{equation*}

	Since $P$ is a polynomial, we have that \[\left| N_{ij }^{} - \tilde{N_{ij }^{}} \right| \leq C \left| M_{ij }^{} - \tilde{M_{ij }^{}} \right|\] where $C$ is a constant depending only on $\max \left| M_{ij }^{}  \right| $ if we keep $\epsilon , \delta \leq 1$. Hence as $t_0 \leq \delta $,

	\begin{equation}
	\begin{split}
		N_{ij }^{} X^i X^j 
		& \geq -C \left| M_{ij }^{} - \tilde{M_{ij }^{}} \right|\\
		&= -C \left| \epsilon (\delta + t_0) g_{ij }^{}  \right|\\
		& \geq -2C \epsilon \delta.
	\end{split}
	\end{equation}

	Let $f=\tilde{M}_{ij }^{} X^i X^j$. Observe that $f ( x_0,t) > 0 $ for $t<t_0$ and $f ( x_0,t_0) =0$ which imply that $\frac{\partial }{\partial t} f \leq 0$ for $t<t_0$. At $t=t_0$, we see that $f=0$ attains a minimum at $x_0$. Hence $\nabla f=0$ and $\Delta f \geq 0$ at $( x_0,t_0) $.

	We can extend the vector $X^i$ to a parallel vector field in a neighborhood of $x_0$ along geodesics passing $x_0$ and define $X^i$ on $[0,t_0]$ independent of $t$.
	Then we have that
	
	\begin{align*}
		\frac{\partial }{\partial t} f&=(\frac{\partial }{\partial t} \tilde{M}_{ij }^{}) X^i X^j\\
		\nabla_k f&= (\nabla_k \tilde{M}_{ij }^{}) X^i X^j = (\nabla_k M_{ij }^{} )X^i X^j\\
		\Delta f &= (\Delta  \tilde{M}_{ij }^{}) X^i X^j = (\Delta M_{ij }^{} )X^i X^j
	\end{align*}
	
	Therefore,
	\begin{equation*}
	\begin{split}
		\frac{\partial }{\partial t} f
		&=(\frac{\partial }{\partial t} \tilde{M}_{ij }^{}) X^i X^j\\
	&= (\frac{\partial }{\partial t} (M_{ij }^{} + \epsilon (\delta +t) g_{ij }^{} )) X^i X^j\\
	&= (\frac{\partial }{\partial t} M_{ij }^{} ) X^i X^j +\epsilon g_{ij }^{} X^i X^j+ \epsilon (\delta +t)(\frac{\partial }{\partial t} g_{ij }^{}) X^i X^j\\
	&= (\frac{\partial }{\partial t} M_{ij }^{} ) X^i X^j +\epsilon\\
	&= \Delta f + u^k \nabla_{k}^{} f +N_{ij }^{} X^i X^j + \epsilon \\
	&= (1-2c \delta ) \epsilon.
	\end{split}
	\end{equation*}
	Then contradiction arises when $2c \delta <1.$ 
\end{proof}

\section{Evolution Equations for Geometric Quantities}

Let $M$ be a compact uniformly convex $n$-dimensional surface smoothly embedded in $\R^{n+1}$. Then $M$ can be represented by the following local diffeomorphism:
\[
	F: U \subset \R^n \rightarrow M \subset \R^{n+1}.
\]
Then the metric $g_{ij}$ and the second fundamental form $h_{ij}$ at $F(\vec{x}) \in M$ can be written as
\[
	g_{ij}(\vec{x})=\left( \frac{\partial F(\vec{x})}{\partial x_i}, \frac{\partial F(\vec{x})}{\partial x_j} \right) \quad h_{ij}(\vec{x})=\left( -\nu (\vec{x}), \frac{\partial^2 F(\vec{x})}{\partial x_i \partial x_j} \right)
\]
where $\left( \cdot , \cdot \right) $ is the standard inner product in $\R^{n+1}$ and $\nu (\vec{x}) \in \R^{n+1}$ is the outward normal to $M$ at $F(\vec{x})$.

Now we denote $M$ by $M_0$ and $F$ by $F_0$. We say a family of maps $F(\cdot , t)$ satisfies the mean curvature flow equation with initial condition $F_0$ if
\begin{align*}
	 & \frac{\partial }{\partial t} F(\vec{x},t)=-H(\vec{x},t) \cdot \nu (\vec{x},t), \quad \vec{x} \in U, \\
	 & F(\cdot ,0)=F_0,
\end{align*}
where $H(\vec{x},t)$ is the mean curvature on $M_t$.

Since the embedding map $F$ is evolving under time $t$, if we fix a point $\vec{x} \in U$, we have that geometric quantities on $F(\vec{x},t) \in M_t$ are also evolving under time $t$. By the evolution equation $\frac{\partial }{\partial t} F(\vec{x},t)=-H(\vec{x},t) \cdot \nu (\vec{x},t)$ for $F$, we can derive evolution equations for other geometric quantities.

\begin{lemma}
	The follows evolving equations hold.
	\begin{enumerate}

		\item $\frac{\partial }{\partial t} g_{ij}=-2Hh_{ij}$
		\item $\frac{\partial }{\partial t} g^{ij}=2Hh^{ij}$
		\item $\frac{\partial \nu }{\partial t} = \nabla H$
		\item $\frac{\partial }{\partial t} h_{ij}=\Delta h_{ij}-2Hh_{ik}g^{kl}h_{lj}+\left| A \right| ^2 h_{ij}$
		\item $\frac{\partial }{\partial t} H=\Delta H+ \left| A \right| ^2 H$
		\item $\frac{\partial }{\partial t} \left| A \right| ^2 = \Delta \left| A \right| ^2 - 2 \left| \nabla A \right| ^2 + 2 \left| A \right| ^4$
	\end{enumerate}
\end{lemma}

\begin{proof}
	\begin{enumerate}
		\item Since $\left( \nu ,\frac{\partial F}{\partial x_{i}}  \right) =0$, by the product rule, we have that
		      \begin{equation*}
			      \begin{split}
				      \frac{\partial }{\partial t} g_{ij}^{}
				      & = \frac{\partial }{\partial t} \left( \frac{\partial F(\vec{x},t) }{\partial x_{i}} , \frac{\partial F(\vec{x},t) }{\partial x_{j }}  \right)                                                                                                               \\
				      & = \left( \frac{\partial }{\partial x_{i}} (-H (\vec{x},t) \cdot \nu (\vec{x},t)) , \frac{\partial F}{\partial x_{j}} ) \right) + \left( \frac{\partial F}{\partial x_{i}}, \frac{\partial }{\partial x_{j}} (-H (\vec{x},t) \cdot \nu (\vec{x},t))  \right) \\
				      & = -H(\left( \frac{\partial \nu }{\partial x_{i}} , \frac{\partial F}{\partial x_{j}}  \right) + \left( \frac{\partial F}{\partial x_{i}} , \frac{\partial \nu }{\partial x_{j}}  \right) )                                                                  \\
				      & = -2H h_{ij}^{}
			      \end{split}
		      \end{equation*}
		\item Since $g_{km}^{} g_{}^{mj} = \delta_{k}^{j} $, we have that
		      \begin{align*}
			      \frac{\partial }{\partial t} (g_{km}^{} g_{}^{mj})                                                  & =0                                \\
			      \frac{\partial g_{km}^{} }{\partial t} g_{}^{mj} + g_{km}^{} \frac{\partial g_{}^{mj} }{\partial t} & =0                                \\
			      -2H h_{km}^{} g_{}^{mj} + g_{km}^{} \frac{\partial g_{}^{mj} }{\partial t}                          & =0                                \\
			      g_{}^{ik} g_{km}^{} \frac{\partial g_{}^{mj} }{\partial t}                                          & =g_{}^{ik} 2H h_{km}^{} g_{}^{mj} \\
			      \frac{\partial }{\partial t} g_{}^{ij}                                                              & = 2H h_{}^{ij}.
		      \end{align*}
		\item Since $\left| \nu \right| =1$ is fixed, we have that $\frac{\partial \nu }{\partial t} $ lies in the tangent space of the surface. Hence we can assume that $\frac{\partial \nu }{\partial t} = V^i \frac{\partial F}{\partial x_{i}} \in \R^{n+1}$ where $V^i$ can be determined by the following identity \[\left( \frac{\partial \nu }{\partial t} ,\frac{\partial F}{\partial x_{j}}  \right) = g_{ij}^{} V^i\]. Thus, we have that
		      \begin{equation*}
			      \begin{split}
				      \frac{\partial \nu }{\partial t} &=g_{}^{ij} \left( \frac{\partial \nu }{\partial t} ,\frac{\partial F}{\partial x_{j}}  \right) \cdot \frac{\partial F}{\partial x_{i}} \\
				      &= -g_{}^{ij} \left( \nu  ,\frac{\partial }{\partial t} \frac{\partial F}{\partial x_{j}}  \right) \cdot \frac{\partial F}{\partial x_{i}} \\
				      &= g_{}^{ij} \left( \nu  , \frac{\partial }{\partial x_{j}} (H (\vec{x},t) \cdot \nu (\vec{x},t) ) \right) \cdot \frac{\partial F}{\partial x_{i}} \\
				      &= g_{}^{ij} \frac{\partial }{\partial x_{j}} H \frac{\partial F}{\partial x_{i}} \\
				      &=\nabla H
			      \end{split}
		      \end{equation*}
		\item By the Gauss-Weingarten relations, we have that
		      \[
			      \begin{cases}
				      \frac{\partial ^2 F}{\partial x_{i} \partial x_{j}} = \Gamma_{\ ij}^{k} \frac{\partial F}{\partial x_{k}} -h_{ij}^{} \nu \\
				      \frac{\partial \nu }{\partial x_{j}} =h_{jl}^{} g_{}^{lm } \frac{\partial F}{\partial x_{m}} .
			      \end{cases}
		      \]
		      Hence
		      \begin{equation*}
			      \begin{split}
				      \frac{\partial }{\partial t} h_{ij}^{}  &= -\frac{\partial }{\partial t} \left( \nu , \frac{\partial^2 F}{\partial x_{i} \partial x_{j}} \right)  \\
				      &= -\left( g_{}^{pq } \frac{\partial }{\partial x_{p}} H \frac{\partial F}{\partial x_{q}} , \frac{\partial^2 F}{\partial x_{i} \partial x_{j}}  \right) + \left( \nu , \frac{\partial^2 }{\partial x_{i} \partial x_{j}} (H \cdot \nu ) \right) \\
				      &= -\left( g_{}^{pq } \frac{\partial }{\partial x_{p}} H \frac{\partial F}{\partial x_{q}} , \Gamma_{\ ij}^{k} \frac{\partial F}{\partial x_{k}} -h_{ij}^{} \nu  \right) + \frac{\partial }{\partial x_{j}} \left( \nu , \frac{\partial }{\partial x_{i}} (H \cdot \nu ) \right) - \left( h_{jl}^{} g_{}^{lm } \frac{\partial F}{\partial x_{m}}, \frac{\partial }{\partial x_{i}} (H \cdot \nu ) \right) \\
				      &= -g_{}^{pq } \frac{\partial H}{\partial x_{q}} \Gamma_{\ ij}^{k} g_{pk}^{} + \frac{\partial^2 H}{\partial x_{i} \partial x_{j}} -H \cdot \left( h_{jl}^{} g_{}^{lm } \frac{\partial F}{\partial x_{m}}, h_{il'}^{} g_{}^{l'm' } \frac{\partial F}{\partial x_{m'}} \right) \\
				      &= \frac{\partial^2 H}{\partial x_{i} \partial x_{j}} - \Gamma_{\ ij}^{q} \frac{\partial H}{\partial x_{q}} - H h_{j}^{m} h_{i}^{n} g_{mn }^{}
			      \end{split}
		      \end{equation*}
		      Since $H$ is a scalar function, we have that \[\nabla_{i}\nabla_{j} H = \frac{\partial^2 H}{\partial x_{i} \partial x_{j}} - \Gamma_{\ ij}^{q} \frac{\partial H}{\partial x_{q}}\] where $\nabla $ is the Levi-Civita connection on $M_t$. From previous lemma, we have the Simon's identity that \[\Delta h_{ij}^{} = \nabla_{i}\nabla_{j} H +H h_{li }^{} g_{}^{lm } h_{mj}^{} - \left| A \right| ^2 h_{ij}^{} .\] Hence
		      \begin{equation*}
			      \begin{split}
				      \frac{\partial }{\partial t} h_{ij}^{}
				      &= \frac{\partial^2 H}{\partial x_{i} \partial x_{j}} - \Gamma_{\ ij}^{q} \frac{\partial H}{\partial x_{q}} - H h_{j}^{m} h_{i}^{n} g_{mn }^{} \\
				      &= \Delta h_{ij}^{} -( H h_{li }^{} g_{}^{lm } h_{mj}^{} - \left| A \right| ^2 h_{ij}^{} ) - \Gamma_{\ ij}^{q} \frac{\partial H}{\partial x_{q}} - H h_{j}^{m} h_{i}^{n} g_{mn }^{}\\
				      &= \Delta h_{ij}^{} - 2 H h_{li }^{} g_{}^{lm } h_{mj}^{} + \left| A \right| ^2 h_{ij}^{}.
			      \end{split}
		      \end{equation*}

		\item Since $H=g_{}^{ij} h_{ij}^{} $, we have that
		      \begin{equation*}
			      \begin{split}
				      \frac{\partial }{\partial t} H= \frac{\partial }{\partial t} (g_{}^{ij} h_{ij}^{} )&=\frac{\partial g_{}^{ij} }{\partial t} h_{ij}^{} + g_{}^{ij} \frac{\partial h_{ij}^{} }{\partial t} \\
				      &=2H h_{}^{ij} h_{ij}^{} + g_{}^{ij} (\Delta h_{ij}^{} - 2 H h_{li }^{} g_{}^{lm } h_{mj}^{} + \left| A \right| ^2 h_{ij}^{})\\
				      &=\Delta H + \left| A \right| ^2 H.
			      \end{split}
		      \end{equation*}

		\item Combining previous results, we can deduce the following evolution equation \begin{equation*}
			      \begin{split}
				      \frac{\partial }{\partial t} h_{i}^{\ j}
				      &= \frac{\partial }{\partial t} (h_{ik}^{} g_{}^{kj} )\\
				      &= (\Delta h_{ik}^{} - 2 H h_{li }^{} g_{}^{lm } h_{mk}^{} + \left| A \right| ^2 h_{ik}^{})g_{}^{kj}+ h_{ik}^{} (2H h_{}^{kj} )\\
				      &= \Delta h_{i}^{\ j} - 2H h_{ik}^{} h_{}^{kj} + \left| A \right| ^2 h_{i}^{\ j} - 2H h_{ik}^{} h_{}^{kj}\\
				      &= \Delta h_{i}^{\ j} + \left| A \right| ^2 h_{i}^{\ j}.
			      \end{split}
		      \end{equation*}
		      Since $\left| A \right| ^2=h_{}^{ij} h_{ij}^{} = h_{i}^{\ j} h_{\ j}^{i}$, we have that
		      \begin{equation*}
			      \begin{split}
				      \frac{\partial }{\partial t} \left| A \right| ^2
				      &= \frac{\partial }{\partial t}  (h_{i}^{\ j} h_{\ j}^{i}) \\
				      &= (\Delta h_{i}^{\ j} + \left| A \right| ^2 h_{i}^{\ j})h_{\ j}^{i} + h_{i}^{\ j}(\Delta h_{\ j}^{i} + \left| A \right| ^2 h_{\ j}^{i})\\
				      &= 2(h_{}^{ij} \Delta h_{ij}^{} + \left| A \right| ^4)\\
			      \end{split}
		      \end{equation*}
		      Since the connection $\nabla $ is compatible with the metric $g$, we have that
		      \begin{equation*}
			      \begin{split}
				      \Delta \left| A \right| ^2
				      &= g_{}^{mn } \nabla_{m}\nabla_{n} (h_{}^{ij} h_{ij}^{})  \\
				      &= 2g_{}^{mn } \nabla_{m} (h_{}^{ij} \nabla_{n}h_{ij}^{})  \\
				      &= 2(g_{}^{mn } \nabla_{m}\nabla_{n}h_{ij}^{}) h_{}^{ij} + 2g_{}^{mn } (\nabla_{m} h_{}^{ij}) (\nabla_{n}h_{ij}^{}) \\
				      &= 2 h_{}^{ij} \Delta h_{ij}^{} + 2 \left| \nabla A \right| ^2.
			      \end{split}
		      \end{equation*}
		      It follows that
		      \begin{equation*}
			      \begin{split}
				      \frac{\partial }{\partial t} \left| A \right| ^2
				      &= 2(h_{}^{ij} \Delta h_{ij}^{} + \left| A \right| ^4)\\
				      &= \Delta \left| A \right| ^2 - 2 \left| \nabla A \right| ^2 + \left| A \right| ^4.
			      \end{split}
		      \end{equation*}
	\end{enumerate}
\end{proof}
\section{Preservation of the convexity and pinching condition}
\begin{theorem}
	If $h_{ij}^{} \geq 0 $ at $t=0,$ then it remains so for $0 \leq t < T.$ 
\end{theorem}
\begin{proof}
	We have that \[\frac{\partial }{\partial t} h_{ij}^{} = \Delta h_{ij}^{} - 2 H h_{li }^{} g_{}^{lm } h_{mj}^{} + \left| A \right| ^2 h_{ij}^{}.\]
	Let $M_{ij }^{} = h_{ij }^{} $ and $N_{ij }^{} = \left| A \right| ^2 h_{ij }^{} - 2 H h_{li }^{} g_{}^{lm } h_{mj}^{}.$
	If vector $X^j$ satisfies that $h_{ij }^{} X^j=0$ for all $i$, then
	\[N_{ij }^{} X^j = \left| A \right| ^2 (h_{ij }^{} X^j) - 2 H h_{li }^{} g_{}^{lm } (h_{mj}^{}X^j)=0.\]
	Hence we can apply \autoref{thm:MP2T} to conclude.
\end{proof}
We can in fact prove a stronger version of the theorem above.
\begin{theorem}
	If $\epsilon H g_{ij }^{} \leq h_{ij }^{} \leq \beta H g_{ij }^{} ,$ and $H \geq 0$ at $t=0,$ then it remains true for $t>0$.
\end{theorem}
\begin{proof}
	First, since $\frac{\partial }{\partial t} H=\Delta H+ \left| A \right| ^2 H$, by \autoref{thm:MPP} we have that if $H \geq 0$ at $t=0$, $H \geq 0$ for all $t \geq 0$.
	Let $M_{ij }^{} = h_{ij }^{} - \epsilon H g_{ij }^{} .$ Then
	\begin{equation*}
	\begin{split}
		\frac{\partial }{\partial t} M_{ij }^{}  
	&= \frac{\partial }{\partial t} h_{ij }^{} -\epsilon (\frac{\partial }{\partial t} H) g_{ij }^{} - \epsilon H \frac{\partial }{\partial t} g_{ij }^{}  \\
	&= \Delta h_{ij}^{} - 2 H h_{li }^{} g_{}^{lm } h_{mj}^{} + \left| A \right| ^2 h_{ij}^{}-\epsilon g_{ij }^{} (\delta H+ \left| A \right| ^2 H)-\epsilon H(-2H h_{ij }^{} )\\
	&= \Delta M_{ij }^{} + \left| A \right| ^{2}h_{ij }^{} + 2 \epsilon H^2 h_{ij }^{} - \epsilon \left| A \right| ^2 H g_{ij }^{} - 2 H h_{li }^{} g_{}^{lm } h_{mj}^{}.
	\end{split}
	\end{equation*}
	Let $N_{ij }^{} = \left| A \right| ^{2}h_{ij }^{} + 2 \epsilon H^2 h_{ij }^{} - \epsilon \left| A \right| ^2 H g_{ij }^{} - 2 H h_{li }^{} g_{}^{lm } h_{mj}^{}$. From direct computation we have that 
	\begin{equation*}
	\begin{split}
		N_{ij }^{} 
		&= \left| A \right| ^2(h_{ij }^{} -\epsilon H g_{ij }^{} )-2H(h_{li }^{} g_{}^{lm } h_{mj}^{}-\epsilon H h_{ij }^{} )\\
		&= \left|A \right| ^2 M_{ij }^{} -2H(h_{li }^{} g_{}^{lm } h_{mj}^{}-\epsilon H h_{li }^{} g_{}^{lm} g_{mj }^{} )\\
		&= \left|A \right| ^2 M_{ij }^{} -2H h_{i}^{m} (h_{mj }^{}-\epsilon H  g_{mj }^{} )\\
		&= \left|A \right| ^2 M_{ij }^{} -2H h_{i}^{m} M_{ mj}^{}. \\
	\end{split}
	\end{equation*}  
	Then for the null vector $X^i$ of $M_{ij }^{} $, we have that
	\begin{equation*}
	\begin{split}
		N_{ij }^{} X^j 
	&= \left|A \right| ^2 (M_{ij }^{}X^j) -2H h_{i}^{m} (M_{ mj}^{} X^j)=0.\\
	\end{split}
	\end{equation*}
	Then the result follows from \autoref{thm:MP2T}.
\end{proof}

\section{Stampacchia's iteration}
The key idea in Huisken's proof is to show the boundedness of certain geometric quantities along the flow where the maximum principle is not applicable. A general approach named Stampacchia's iteration is used to reach the goal. In this section, we introduce the general idea for Stampacchia's iteration.
The essential idea behind Stampacchia's iteration is the following algebraic lemma:
\begin{lemma}
	Let $f : [\bar{x},\infty) \to \R$ be a non-negative and non-increasing function. Suppose for $C>0, p>0$ and $\gamma >1,$ 
	\[ (y-x)^{p}f(y) \leq Cf(x)^{\gamma }, \quad y \geq x \geq \bar{x}.\] 
	Then $f(y)=0$ for $y \geq \bar{x} + d$ where $d^p=C f(\bar{x})^{\gamma -1}2^{\frac{p \gamma }{\gamma -1}}$ 
\end{lemma}
\begin{proof}
	Without loss of generality, we can assume that $\bar{x}=0.$ 
	Let $g=(\frac{f}{f(0)})^{\frac{1}{p}}$ and $A=(Cf(0)^{\gamma -1})^{\frac{1}{p}}.$ For $y \geq x \geq 0,$ we have that
	\begin{align*}
		(y-x)^{p}f(y) &\leq Cf(x)^{\gamma }\\
		A^p (y-x)^{p}f(y) &\leq A^p Cf(x)^{\gamma }\\
		(y-x)^p g(y)^p f(0)^{\gamma } &\leq C f(0)^{\gamma-1 }g(x)^{p \gamma } f(0)^{\gamma }\\
		(y-x)g(y) &\leq A g(x)^{\gamma }.
	\end{align*}
	Now fix $y>0$, let $x_n=y(1-\frac{1}{2^n})$. Note that $\lim_{n \to \infty} x_n=y$ and $x_0=0.$ Hence, we have that $g(x_0)=g(0)=1$ and 
	\begin{align*}
		(x_{n+1}-x_n)g(x_{n+1}) &\leq Ag(x_{n}^{} )^{\gamma }\\
		y(\frac{1}{2^{n}}-\frac{1}{2^{n+1}})g(x_{n+1}^{} ) &\leq Ag(x_{n}^{} )^{\gamma }\\
		g(x_{n+1}^{} ) &\leq \frac{A}{y}2^{n+1} g(x_{n}^{} )^{\gamma }.
	\end{align*} 
	Using the above inequality inductively, we have that \[g(x_{n}^{} ) \leq (\frac{A}{y})^{1+\gamma + \dots + \gamma ^{n-1}} 2^{n+(n-1)\gamma + (n-2)\gamma ^2 + \dots + \gamma ^{n-1}}.\]
	Since \[n+(n-1)\gamma + (n-2)\gamma ^2 + \dots + \gamma ^{n-1}=\frac{\gamma ^n+n-(n+1)\gamma }{(\gamma -1)^2},\]
	if we choose $y$ such that $\frac{A}{y}=2^{-\frac{\gamma }{\gamma -1}}$, then we have that 
	\begin{equation*}
	\begin{split}
		g(x_{n}^{} ) 
	&\leq  (\frac{A}{y})^{\frac{\gamma ^n-1}{\gamma -1}} 2^{\frac{\gamma ^n+n-(n+1)\gamma }{(\gamma -1)^2}} \\
	&\leq 2^{\frac{1}{(\gamma -1)^2}(-\gamma (\gamma ^n-1)+\gamma ^{n+1}+n-(n+1)\gamma )}\\
	&=2^{-\frac{n}{\gamma -1}}.
	\end{split}
	\end{equation*}
	It follows that $\lim_{n \to \infty} g(x_n)=0$. By continuity of $g$, we have that $g(y)=0.$ Therefore, $f(y)=0.$  
\end{proof}


\end{document}